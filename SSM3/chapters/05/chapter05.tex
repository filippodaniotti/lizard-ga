\documentclass[class=book, crop=false, oneside, 12pt]{standalone}
\usepackage{standalone}
\usepackage{../../style}
% \usepackage{../../music-tools}
\graphicspath{{./assets/images/}}


% \provideboolean{isCompiledFromMain}
% \setboolean{isCompiledFromMain}{false}


% arara: lualatex
% arara: biber
% arara: latexmk: { clean: partial }
\begin{document}
\chapter*{Conclusioni}
\addcontentsline{toc}{chapter}{Conclusioni}

In questo lavoro abbiamo presentato una trattazione sintetica dell'esperienza musicale dei Pink Floyd e in particolare dell'album \acrlong{anm}. In Cap.\ref{ch:01-pinkfloyd} abbiamo fornito una panoramica storica completa dei Pink Floyd, partendo dalla loro formazione attraverso le varie ere; abbiamo inoltre fornito un profilo di ogni membro, evidenziando i loro contributi musicali e stili; abbiamo infine concluso con un'analisi del loro profilo musicale. Una sezione speciale si concentra sulla strumentazione di Richard Wright.

In Cap.\ref{ch:02-animals} ci siamo concentrati specificamente sull'album \acrshort{anm}, dettagliando la sua concezione, il processo di registrazione, contenuto tematico e copertina. Il capitolo fornisce un'analisi dettagliata dei testi di ogni traccia, approfondendo il profondo significato metaforico d'ispirazione orwelliana in riferimento al contesto socio-culturale della società britannica degli anni '70.

Nei Capp.\ref{ch:03-pigs} e \ref{ch:04-sheep} abbiamo presentato un'analisi musicale dettagliata della coppia di brani \acrlong{pw1} e \acrlong{pw2} e \acrlong{s}, rispettivamente. Per ciascuno abbiamo presentato un'analisi dettagliata della struttura del brano, approfondendo note di composizione, scrittura e arrangiamento. Infine, abbiamo presentato un'analisi armonica dettagliata, esaminando le progressioni di accordi e gli elementi modali attraverso le diverse sezioni del brano.

Dall'analisi svolta non ho potuto fare a meno di notare che una delle caratteristiche più notevoli di quest'opera è il contrasto tra l'accessibilità dei brani e la loro effettiva complessità, costituita da piccoli elementi che ben si incastrano fra loro e restituiscono un'esperienza di ascolto che riesce a essere sia immediata, sia profonda e stratificata. Sotto quest'aspetto,  sono da considerarsi brani emblematici \acrshort{pw1} e \acrshort{pw2}, sui quali sono state spese oltre 10 pagine di analisi nonostante la durata di poco superiore al minuto.

Dall'analisi di \acrshort{s}, invece, emerge che il suo tratto distintivo è il suo sviluppo graduale: nonostante la considerevole durata, il brano si discosta dalla tipica abbondanza di sezioni fin troppo eterogenee caratteristica delle suite progressive rock. Al contrario, si sviluppa attraverso un numero limitato di sezioni che si sviluppano in maniera graduale, prendendosi il proprio tempo. Ne risulta un brano lungo, articolato, eterogeneo ma allo stesso tempo coerente e organico.

Aspetto ancora più, emerso in entrambe le analisi,  è la perfetta integrazione tra l'elemento compositivo e quello lirico. \acrlong{anm} non è semplicemente un insieme di canzoni con un testo che sta bene sulla progressione di accordi: è un disco in cui la musica  amplifica e sottolinea i significati di un impianto lirico profondo attraverso scelte compositive mirate. Certo, questa sinergia tra parole e musica è resa possibile dal fatto che l'album è il prodotto di un singolo personaggio all'interno di quello che dovrebbe essere un gruppo, e questo elemento autoreferenziale potrebbe risultare indigesto a chi apprezza i contributi compositivi di Gilmour, Wright e Mason. Ma nonostante questo, il risultato è un album che va oltre la somma delle sue parti, che potremmo quasi definire olistico e che, di certo, costituisce un'esperienza artistica di primissimo livello. 

Il valore musicale dei Pink Floyd va ben oltre i numeri dei dati di vendita.

\end{document}