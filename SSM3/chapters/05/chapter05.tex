\documentclass[class=book, crop=false, oneside, 12pt]{standalone}
\usepackage{standalone}
\usepackage{../../style}
% \usepackage{../../music-tools}
\graphicspath{{./assets/images/}}


% \provideboolean{isCompiledFromMain}
% \setboolean{isCompiledFromMain}{false}


% arara: lualatex
% arara: biber
% arara: latexmk: { clean: partial }
\begin{document}
\chapter*{Conclusioni}
\addcontentsline{toc}{chapter}{Conclusioni}

In questo lavoro abbiamo presentato una trattazione sintetica dell'esperienza musicale dei Pink Floyd e in particolare dell'album \acrlong{anm}. In Cap.\ref{ch:01-pinkfloyd} abbiamo fornito una panoramica storica completa dei Pink Floyd, partendo dalla loro formazione attraverso le varie ere; abbiamo inoltre fornito un profilo di ogni membro, evidenziando i loro contributi musicali e stili; abbiamo infine concluso con un'analisi del loro profilo musicale. Una sezione speciale si concentra sulla strumentazione di Richard Wright.

In Cap.\ref{ch:02-animals} ci siamo concentrati specificamente sull'album \acrshort{anm}, dettagliando la sua concezione, il processo di registrazione, contenuto tematico e copertina. Il capitolo fornisce un'analisi dettagliata dei testi di ogni traccia, approfondendo il profondo significato metaforico di ispirazione orwelliana in riferimento al contesto socio-culturale della società britannica degli anni '70.

Nei Capp.\ref{ch:03-pigs} e \ref{ch:04-sheep} abbiamo presentato un'analisi musicale dettagliata della coppia di brani \acrlong{pw1} e \acrlong{pw2} e \acrlong{s}, rispettivamente. Per ciascuno abbiamo presentato un'analisi dettagliata della struttura del brano, approfondendo note di composizione, scrittura e arrangiamento. Infine, abbiamo presentato un'analisi armonica dettagliata, esaminando le progressioni di accordi e gli elementi modali attraverso le diverse sezioni del brano.


\end{document}