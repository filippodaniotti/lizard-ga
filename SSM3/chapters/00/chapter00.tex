% arara: lualatex
% arara: biber
% arara: latexmk: { clean: partial }
\documentclass[class=book, crop=false, oneside, 12pt]{standalone}
\usepackage{standalone}
\usepackage{../../style}

\begin{document}
\chapter*{Prefazione}
\addcontentsline{toc}{chapter}{Prefazione}

Questo testo costituisce la guida all'ascolto per l'esame di Pianoforte Moderno di livello SSM3 presso l'Accademia Musicale Lizard, laboratorio di Mogliano Veneto; l'elaborato completa l'esame svolto in data 6 luglio 2022. La guida si concentra sull'analisi di \emph{\acrlong{anm}} dei Pink Floyd, e in particolare sui brani \acrlong{pw1}, \acrlong{pw2} e \acrlong{s}.

% \paragraph{Software}
Il documento è stato sviluppo e prodotto con LaTeX, in particolare il dialetto LuaLaTeX, per quanto riguarda la parte testuale e la quasi totalità degli asset grafici. GLi spartiti sono stati realizzati con il software di intavolatura LilyPond, per la conveniente integrazione con LaTeX attraverso il pacchetto \texttt{lyluatex}. Gli spettrogrammi sono stati realizzati con le API del pacchetto \texttt{librosa}\footnote{https://librosa.org/doc/latest/index.html}; i parametri usati sono FFT size 2048, window size 2048, hop size 512 e Hanning window; per la trasformazione in scala mel sono stati usati 128 bin di frequenza. Il codice che ha prodotto questo documento è open source e disponibile in licenza Creative Commons CC BY-NC-ND 4.0\footnote{https://creativecommons.org/licenses/by-nc-nd/4.0/legalcode} su GitHub al seguente URL: \url{https://github.com/filippodaniotti/lizard-ga/SSM3}.

% \paragraph{Notazione}
La notazione usata per gli accordi mischia quella proposta in~\cite{brachi2008armonia}, prevalente, e quella adottata da LilyPond~\cite{res:lily-chord-chart}. Questo compromesso è stato reso necessario per evitare ulteriori mal di testa con l'inutilmente complessa infrastruttura software adottata per la realizzazione di questo lavoro.

% \paragraph{Nota deontologica sull'uso di AI}
In questo lavoro sono stati impiegati di conversational agents basati su Large Language Models (LLM), in particolare \emph{Claude 3.5 Sonnet}\footnote{https://www.anthropic.com/news/claude-3-5-sonnet} in assistenza ad alcuni compiti specifici. in particolare, sono stati impiegati per compiti quali la generazione di codice boilerplate LaTeX e LilyPond, traformazioni di struttura del testo (ad esempio, trasformare un elenco in prosa in un elenco puntato e viceversa), e il debugging di problemi legati all'infrastruttura software. Per principio etico e deontologico, non sono stati impiegati LLM per compiti legati alla produzione di contenuti originali. Il contributo degli LLM sulla formulazione delle analisi musicali e la loro articolazione testuale, pertanto, è nullo.

% \paragraph{Struttura}
L'elaborato è strutturato come segue. Il Cap.\ref{ch:01-pinkfloyd} fornisce una panoramica storica dei Pink Floyd, analizzandone la formazione, i membri e il profilo musicale. Il Cap.\ref{ch:02-animals} si concentra sull'album \acrlong{anm}, esaminandone la concezione, la registrazione e un'analisi dettagliata dei testi. Il Cap.\ref{ch:03-pigs} presenta un'analisi musicale approfondita dei brani \acrlong{pw1} e \acrlong{pw2}, mentre il Cap.\ref{ch:04-sheep} è dedicato all'analisi musicale di \acrlong{s}.


\end{document}