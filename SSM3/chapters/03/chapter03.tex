% arara: lualatex: { shell: true }
% arara: latexmk: { clean: partial }
\documentclass[class=book, crop=false, oneside, 12pt]{standalone}
\usepackage{standalone}
\usepackage{../../style}
\usepackage{../../glossary}

\let\isCompiledFromMain\undefined
\usepackage{../../music-tools}

\graphicspath{{./assets/images/}}
\ifdefined\isCompiledFromMain
\else
    \setluaoption{ly}{includepaths}{./assets/scores/}
\fi


\begin{document}
    \chapter{Pigs on the Wing}
    
    % \begin{sheet}{Accordi}
    %     \lilypondfile{./potw-chords.ly}
    % \end{sheet}

    % \chord{t}{3,2,o,o,3,3}{\writechord{g,7}}

    \section{Introduzione}
    brano diviso in due parti chiusura apertura disco
    vanno più intesi come due atti di uno stesso brano, o due versioni, con testi complementari e un medesimo arrangiamento ed esecuzione con sottili variazioni
    cornice narrativa e morale del disco, pensata per lasciare uno spiraglio di luce in chiusura di un'opera altrimenti molto cupa
    ballad acustica, una sorta di poesia cantata

    \section{Struttura}
    A causa ella durata estremamente ridotta, il brano non presenta una struttura scomponibile in sezioni. Di fatto, a ogni verso del testo corrisponde uno sviluppo armonico e melodico diverso, con la sola eccezione dei primi due versi, che sono identici. 

    \subsection{Versioni dell'album}

    \subsubsection*{Introduzione\\ \small{da 0:00 a 0:16}}
    Il brano si apre con una breve introduzione senza troppi fronzoli in cui la chitarra esegue un breve vamp in strumming, rendendo subito l'idea di un brano semplice e diretto, quasi spartano. A 0:15 abbiamo l'ingresso della voce.
    
    \subsubsection*{Scenario\\ \small{da 0:17 a 0:46}}
    \begin{displayquote}
        If you didn't care what happened to me \\
        And I didn't care for you \\ 
    \end{displayquote}
    In questa sezione Waters sta descrivendo l'ípotetica situazione tra sé stesso e l'interlocutore. In \acrshort{pw1} si parla di cosa succederebbe se gli esseri umani non si interessassero gli uni degli altri, mentre invece \acrshort{pw2} descrive lo scenario opposto. Questi due versi sono gli unici la cui progressione armonica viene ripetuta più di una volta, costituendo di fatto la sezione più lunga del brano. Le ragioni di questa scelta sono probabilmente da ricercare nella volontà di dare un punto di partenza stabile per il brano; se infatti l'armonia fosse cambiata già al secondo verso, il brano sarebbe risultato troppo instabile e sarebbe mancato di compiutezza.

    \subsubsection*{Conseguenze\\ \small{da 0:47 a 1:13}}
    \begin{displayquote}
        We would zig-zag our way through the boredom and pain \\
        Occasionally glancing up through the rain \\
        Wondering which of the buggers to blame \\
    \end{displayquote}
    In questa sezione la progressione armonica è unica per ogni verso ed è pertanto in pieno supporto alla parte vocale e al testo di Waters. Il narratore qui sta descrivendo le conseguenze dello scenario descritto nei due versi precedenti: in \acrshort{pw1} abbiamo un'umanità che versa in uno stato di profonda infelicità e, al contempo, incapacità di individuarne la causa; in \acrshort{pw2}, al contrario, vediamo un narratore sereno, non più preda dell'incertezza.

    
    \subsubsection*{Finale\\ \small{da 1:14 a 1:24}}
    \begin{displayquote}
        And watching for pigs on the wing
    \end{displayquote}
    Nel finale la chitarra indugia su un accordo di dominante nell'attesa che il narratore riveli l'entità che, nella simbologia dell'album, raccoglie e sintetizza il malessere e l'inquetudine: i maiali volanti. A seguito di questa realizzazione, il vamp dell'introduzione viene ripreso, per terminare con uno strum leggermente sgranato dell'accordo di primo grado.

    \subsection{Versione Stereo8}
    La versione stereo8 del brano è sostanzialmente una concatenazione delle due parti nella loro versione dell'album, conl'intermezzo dell'assolo registrato da Snowy White. 

    \begin{itemize}
        \item \emph{da 0:00 a 1:24}: \acrlong{pw1};
        \item \emph{da 1:25 a 2:10}: assolo di Snowy White;
        \item \emph{da 2:10 a 3:26}: \acrlong{pw2};
    \end{itemize}

    \section{Tecnica e arrangiamento}
    arrangiamento chitarra e voce, cfr if, wish you were here, qui nella sua forma più essenziale
    accompagnamento di chitarra strumming, accordi aperti, nessun arpeggio né temi
    grande attenzione all'elemento ritmico, ambiguità nel metro impiegato e grande variabilità
    metro musicale in piena funzione del testo
    in part 1 claudicante, con misure tagliate, in part 2 più regolare, più stabile come loro

    solo di snowy white...

    \section{Analisi armonica}
    il brano apre in sol
    cadenza plagale I-IV, ripresa della struttura blues
    Ingresso della voce va sul IV e usa anche lì una versione soft della plagale, in analogia con l'introduzione, siamo comunque in sol nonstante il fa naturale
    risolve tornando al pattern dell'intro
    a zig zag fa un la , buttando anche qua il sus4 sempre analogia, e poi a7
    non sembra una modulazione che ti porta lontano, non sembra nemmeno intenzionale, sembra fatto per sbaglio, come loro che vanno a zig zag
    infatti subito dopo torna in sol e fa I-IV-V
    invece dopo si allontana: va in Am e fa VIm-bVI-VIm-bVI 
    va sul I, ma subito dopo ricade nel VIm
    da qui va in D7, dominante secondaria per risolvere in sol
    tipo modulazione indiretta, fa il ii-V di sol per toranre in sol



\end{document}