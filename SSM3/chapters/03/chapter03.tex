% arara: lualatex: { shell: true }
% arara: latexmk: { clean: partial }
\documentclass[class=book, crop=false, oneside, 12pt]{standalone}
\usepackage{standalone}
\usepackage{../../style}
\usepackage{../../glossary}

\let\isCompiledFromMain\undefined
\usepackage{../../music-tools}

\graphicspath{{./assets/images/}}
\ifdefined\isCompiledFromMain
\else
    \setluaoption{ly}{includepaths}{./assets/scores/}
\fi


\begin{document}
    \chapter{Pigs on the Wing}
    La scelta di questo brano per una guida all'ascolto può sembrare singolare e forse anche un po' bislacca. Pigs on the Wing non è un brano particolarmente interessante né dal punto di vista compositivo, né dal punto di vista dell'arrangiamento, né tantomeno dal punto di vista tecnico. È una semplice ballad di poco più di un minuto, che fa da introduzione e chiusura a tre brani di tutt'altro spessore sotto qualsiasi parametro musicale.

    Tuttavia, il diavolo è nei dettagli e se le due parti di Pigs on the Wing possono apparire a fianco di brani come \acrshort{d}, \acrshort{p} e \acrshort{s}, è perché funzionano bene. Non solo: si tratta di un brano unico all'interno della discografia dei Floyd, perché è l'unico in cui ogni elemento compositivo - armonia, arrangiamento, metro - si piegano completamente in funzione dell'elemento lirico. Questo fatto è la chiave che useremo per analizzare l'intero brano e ci auguriamo che, alla fine di questo capitolo, le motivazioni che rendono Pigs on the Wing meritevole di un'analisi che vada oltre dei tre minuti scarsi necessari all'ascolto siano chiare anche al lettore.

    In Sec.\ref{sec:03-intro} introdurremo il brano e ne presenteremo le generalità. In Sec.\ref{sec:03-structure} analizzeremo la struttura del brano e ne proporremo una suddivisione in sezioni. In Sec.\ref{sec:03-arrangement} analizzeremo l'arrangiamento e dettagli dell'esecuzione e produzione. Infine, in Sec.\ref{sec:03-harmony} analizzeremo l'armonia del brano e ne presenteremo una breve analisi.

    \section{Introduzione}
    \label{sec:03-intro}
    \acrfull{pw1} e \acrfull{pw2} rappresentano rispettivamente l'apertura e la chiusura di \acrshort{anm}. Nonostante la presenza delle denominazioni \gquotes{Part 1} e \gquotes{Part 2} sottintendere una sequenzialità cronologica tra i due brani, questi andrebbero più considerati come due atti di uno stesso brano: due versioni con testi complementari e un arrangiamento che, al netto di sottilissime variazioni, è il medesimo. In luce di questo, salvo osservazioni specifiche a una delle due parti, ci riferiremo ai due brani indistintamente, come fossero un'unica composizione.

    Pigs on the Wing è una ballad acustica scritta da Roger Waters della durata di poco meno di un minuto e mezzo. L'arrangiamento è essenziale e ridotto esclusivamente alla voce di Waters accompagnata da una chitarra acustica, rendendolo di fatto più simile a una breve poesia cantata che non a una delle altre numerose ballad acustiche scritte dai Floyd. L'ispirazione del contenuto del testo proviene dalla vita privata di Waters e, in particolare, dalla sua relazione con la moglie Carolyne\cite{schaffner1992saucerful}. All'interno dell'album svolge molteplici funzioni: 
    \begin{itemize}
        \item agisce da cornice narrativa e fornisce una chiave interpretativa morale per l'elemento di denuncia sociale presentato nei testi degli altri tre brani;
        \item contrasta e alleggerisce la natura molto cupa e disillusa nel resto del disco; questo è vero soprattutto per \acrshort{pw2}, che chiude l'album con un chiaro messaggio di speranza. Waters avrebbe inseguito affermato che, senza di questo brano, il disco sarebbe stato semplicemente una sorta di grido di rabbia\cite{schaffner1992saucerful};
        \item rappresenta le dimensioni che il potere decisionale di Waters aveva assunto in quel momento: la casa discografica, infatti,  assegnava le royalties sulla base di quante tracce ogni musicista aveva inciso. Quindi, il fatto che Waters abbia potuto un brano in due tracce separate è un forte indicatore che, nel periodo di registrazione di \acrshort{anm}, la sua leadership era incredibilmente solida.
    \end{itemize}
 
    È stata prodotta una versione alternativa del brano in cui le due parti sono riprodotte una di seguito all'altra e inframmezzate da una sezione bridge con un assolo di chitarra eseguito da Snowy White, chitarrista che sarebbe diventato turnista per i Floyd per lo \emph{In The Flesh tour}. Questa versione è stata distribuita inizialmente su formato Stereo8 e, successivamente, in \emph{Goldtop}, album compilation di White uscito nel 1995.
    
    \section{Struttura}
    \label{sec:03-structure}
    A causa ella durata estremamente ridotta, il brano non presenta una struttura scomponibile in sezioni. Di fatto, a ogni verso del testo corrisponde uno sviluppo armonico e melodico diverso, con la sola eccezione dei primi due versi, che sono identici. 
    
    \subsection{Versioni dell'album}
    
    \subsubsection*{Introduzione\\ \small{da 0:00 a 0:16}}
    Il brano si apre con una breve introduzione senza troppi fronzoli in cui la chitarra esegue un breve vamp in strumming, rendendo subito l'idea di un brano semplice e diretto, quasi spartano. A 0:15 abbiamo l'ingresso della voce.
    
    In funzione delle analisi successive, questa sezione sarà denominata \emph{Introduzione}.

    \subsubsection*{Linea 1 - Scenario\\ \small{da 0:17 a 0:46}}
    \begin{displayquote}
        If you didn't care what happened to me \\
        And I didn't care for you \\ 
        \vspace{0pt}
        \rule{.4\linewidth}{.5pt} \\ %\vspace{.5em}
        You know that I care what happens to you \\
        And I know that you care for me 
    \end{displayquote}
    In questa sezione Waters sta descrivendo l'ípotetica situazione tra sé stesso e l'interlocutore. In \acrshort{pw1} si parla di cosa succederebbe se gli esseri umani non si interessassero gli uni degli altri, mentre invece \acrshort{pw2} descrive lo scenario opposto. Questi due versi sono gli unici la cui progressione armonica viene ripetuta più di una volta, costituendo di fatto la sezione più lunga del brano. Le ragioni di questa scelta sono probabilmente da ricercare nella volontà di dare un punto di partenza stabile per il brano; se infatti l'armonia fosse cambiata già al secondo verso, il brano sarebbe risultato troppo instabile e sarebbe mancato di compiutezza.

    In funzione delle analisi successive, questa sezione sarà denominata \emph{Linea 1}.
    
    \subsubsection*{Linee 2 e 3 - Conseguenze\\ \small{da 0:47 a 1:13}}
    \begin{displayquote}
        We would zig-zag our way through the boredom and pain \\
        Occasionally glancing up through the rain \\
        Wondering which of the buggers to blame \\
        \vspace{0pt}
        \rule{.4\linewidth}{.5pt} \\ %\vspace{.5em}
        So I don’t feel alone on the weight of the stone \\
        Now that I’ve found somewhere safe to bury my bone \\
        And any fool knows a dog needs a home 
    \end{displayquote}
    In questa sezione la progressione armonica è unica per ogni verso ed è pertanto in pieno supporto alla parte vocale e al testo di Waters. Il narratore qui sta descrivendo le conseguenze dello scenario descritto nei due versi precedenti: in \acrshort{pw1} abbiamo un'umanità che versa in uno stato di profonda infelicità e, al contempo, incapacità di individuarne la causa; in \acrshort{pw2}, al contrario, vediamo un narratore sereno, non più preda dell'incertezza.
    
    Per comodità consultativa dividiamo questi tre versi in due unità: 
    \begin{itemize}
        \item \emph{Linea 2}: versi \gquotes{We would zig-zag} e \gquotes{Occasionally};
        \item \emph{Linea 3}: verso \gquotes{Wondering}.
    \end{itemize}
        
        
    \subsubsection*{Finale\\ \small{da 1:14 a 1:24}}
    \begin{displayquote}
        And watching for pigs on the wing \\
        \vspace{0pt}
        \rule{.4\linewidth}{.5pt} \\ %\vspace{.5em}
        A shelter from pigs on the wing 
    \end{displayquote}
    Nel finale la chitarra indugia su un accordo di dominante nell'attesa che il narratore riveli l'entità che, nella simbologia dell'album, raccoglie e sintetizza il malessere e l'inquetudine: i maiali volanti. A seguito di questa realizzazione, il vamp dell'introduzione viene ripreso, per terminare con uno strum leggermente sgranato dell'accordo di primo grado.
    
    In funzione delle analisi successive, questa sezione sarà denominata \emph{Linea 4}.

    \subsection{Versione Stereo8}
    La versione stereo8 del brano è sostanzialmente una concatenazione delle due parti nella loro versione dell'album, conl'intermezzo dell'assolo registrato da Snowy White. 

    \begin{itemize}
        \item \emph{da 0:00 a 1:24}: \acrlong{pw1};
        \item \emph{da 1:25 a 2:10}: assolo di Snowy White;
        \item \emph{da 2:10 a 3:26}: \acrlong{pw2};
    \end{itemize}

    \section{Tecnica e arrangiamento}
    \label{sec:03-arrangement}

    \subsection{Arrangiamento}
    Per quanto riguarda l'arrangiamento, il brano è essenziale: si tratta di una ballad acustica in cui la chitarra acustica e la voce sono gli unici elementi presenti. La scelta della ballad acustica come stile e struttura compositiva non è un unicum nella produzione discografica dei Floyd: alcuni esempi che possiamo riportare sono If (\acrshort{ahm}), Green Is the Colour (\acrshort{more}), Wish You Were Here (\acrshort{wywh}). Tuttavia, \acrshort{pw1} e \acrshort{pw2} si distinguono fra questi brani. In primo luogo, la durata degli altri brani sopra riportati è quella di una canzone standard (tra i tre e i cinque minuti), mentre \acrshort{pw1} e \acrshort{pw2} sono brani di 1 minuto e mezzo circa. Conseguentemente, a differenza  degli altri brani che presentano una struttura, a differenza del nostro brano che non ha una vera e propria struttura (si veda Sec.\ref{sec:03-structure}). In secondo luogo, i brani sopra riportati hanno un arrangiamento più ricco, con l'aggiunta di altri strumenti (basso, batteria, pianoforte), mentre invece \acrshort{pw1} e \acrshort{pw2} sono eseguiti solo con chitarra acustica e voce. Infine, la parte di chitarra acustica negli altri brani ha una qualche sorta di elaborazione (If è costruita su arpeggi, Wish You Were Here combina un tema con lo strumming degli accordi); in \acrshort{pw1} e \acrshort{pw2}, invece, gli accordi sono eseguiti in semplice strumming. Inoltre, come si vede in Fig.\ref{fig:03-chords}, gli accordi scelti sono tutti diteggiati in prima posizione, o con minime variazioni di essi (le varie sospensioni). Si potrebbe affermare senza grande timore di smentita che \acrshort{pw1} e \acrshort{pw2} sono tra i brani più semplici dal punto di vista esecutivo nella produzione discografica dei Floyd.
    
    \begin{figure}
        \subimport{./assets/figures/}{chord_diagrams.tex}
        \caption{Diagrammi degli accordi e diteggiature usati nel brano.}
        \label{fig:03-chords}
    \end{figure}

    \subsection{Metro}
    Possiamo ragionevolmente ipotizzare che Waters abbia scelto un arrangiamento così essenziale per prevenire ogni possibile distrazione dal testo, vero protagonista di un brano che si puo considerare, a tutti gli effetti, una poesia cantata. Questa necessita è ulteriormente motivata dalla divisione in due atti del brano: per una comprensione piena del messagio che Waters vuole lanciare, e infatti necessario tenere a mente il testo di entrambe le parti. Infine, la durata estremamente ridotta non avrebbe comunque dato spazio a grandi sviluppi strumentali, per cui la scelta di un arrangiamento così semplice, per quanto anticonvenzionale, è ragionevolmente motivata. 

    Bisogna però essere cauti nel bollare questo brano come banale. Se infatti  tecnica e arrangiamento  non risultano particolarmente interessanti, il metro riserva ben altre sorprese. Di primo acchito, si potrebbe dire che il brano è in \meter{4}{4}. Nonostante le prime misure del brano siano perfettamente scomponibili in tale metro, ci possiamo subito rendere conto che il pattern ritmico dello strumming e l'accentazione vanno meglio descritte con una suddivisione ternaria, con una misura di \meter{6}{8} seguita da una di \meter{2}{8}; tuttavia, per mantenere la cellula ritmica e armonica del vamp inalterata, impiegheremo la notazione metrica combinata   \meter{3+3+2}{8}. Nei primi \meter{3+3}{8} viene suonato l'accordo principale della misura, mentre negli ultimi \meter{2}{8} viene suonato un accordo di abbellimento (ne discuteremo più approfonditamente in Sec.\ref{sec:03-harmony}). 
    
    \begin{sheet}[htbp]
        \centering
        \lilypondfile{./potw-rhythm_intro.ly}
        \caption{Pattern ritmico dell'introduzione.}
        \label{sheet:potw-rhythm_intro}
    \end{sheet}

    Questo pattern ritmico viene mantenuto per tutta la prima metà del brano (sezioni Intro e Linea 1), ma viene successivamente abbandonato. Non solo: viene abbandonato anche il metro stesso. In rapida successione, tra Linea 2 e Linea 3, si susseguono \meter{3+3+2}{8}, \meter{3}{4}, \meter{3}{8}, \meter{6}{8} e \meter{5}{8}.

    \begin{sheet}[htbp]
        \centering
        \lilypondfile{./potw1-rhythm_line23.ly}
        \caption{Mentro nelle sezioni Linea 2 e Linea 3 in \acrshort{pw1}.}
        \label{sheet:potw1-rhythm_line23}
    \end{sheet}
    
    La scelta di usare questa notazione per il metro può apparire come un tentativo da parte nostra di far apparire il brano ritmicamente più complesso di come in realtà non sia. Tuttavia, riteniamo che la nostra notazione metrica sia assolutamente coerente con le scelte fatte da Waters e con l'analisi fatta precedentemente su tecnica e arrangiamento. La chiave per comprendere queste scelte è, ancora una volta, la volontà di dare priorità assoluta all'elemento lirico. Relativamente alla tecnica/arrangiamento, questa si declina in un arrangiamento spartano ed essenziale; relativamente al metro, si declina ubvece nel fatto che il metro e le durate delle misure di piegano e si adattano al testo di Waters, e non viceversa. Non solo, ma il metro si adatta anche alla semantica: nel verso \gquotes{Occasionally glancing up through the rain}, l'alzare lo sguardo viene reso musicalmente cambiando il pattern ritmico dal precedente ternario \meter{3+3+2}{8} al binario \meter{3}{4}. Similmente, in \gquotes{Wondering which of the buggers to blame} la misura dispari da \meter{5}{8} rende l'andamento armonico claudicante, a simboleggiare l'affanno e la lotta portata avanti dall'essere umano. A conferma dell'importanza della sinergia tra l'elemento lirico e ritmico, vediamo in Rg.\ref{sheet:potw2-rhythm_line23} che in \acrshort{pw2} molte delle misure tagliate di cui sopra vengono arrotondate al metro pari più vicino e rendono pertanto l'andatura più regolare, in accordo con il contenuto più sereno di \acrshort{pw2}.

    \begin{sheet}[htbp]
        \centering
        \lilypondfile{./potw2-rhythm_line23.ly}
        \caption{Mentro nelle sezioni Linea 2 e Linea 3 in \acrshort{pw2}.}
        \label{sheet:potw2-rhythm_line23}
    \end{sheet}
    
    \subsection{Assolo}
    L'assolo di Snowy White è piuttosto semplice e melodico, privo di virtuosismi: è prevalentemente costruito sulla scala maggiore, utilizza arpeggi e insiste sulle note degli accordi anche per l'intera misura, sfruttando l'articolazione data da bending e vibrati per mantenere il solo interessante. L'elemento più interessante è forse dato dall'utilizzo del Fa naturale nelle prime due misure del solo, sopra al \writechord{C}, in una frase legata. Non è l'unica occorrenza di Fa naturale all'interno di questo brano, ma qui viene utilizzato a fini melodici, in una sezione dove il centro tonale è chiaramente Sol maggiore. In questo senso, l'uso del Fa naturale conferisce a queste frasi un sottile sapore misolidio.
    
    \begin{sheet}[htbp]
        \centering
        \lilypondfile{./potw-phrases_solo.ly}
        \caption{Prime quattro misure dell'assolo.}
        \label{sheet:potw-phrases_solo}
    \end{sheet}

    
    \section{Analisi armonica}
    \label{sec:03-harmony}
    L'introduzione del brano è costituita da un vamp di due accordi, un \writechord{G} e un {\writechord{C},} rispettivamente della durata di \meter{6}{8} e \meter{2}{8}. Questo vamp viene ripetuto per 4 misure (Rg.\ref{sheet:potw-chords_intro}), per poi essere ripreso in chiusura del brano. 
    \begin{sheet}[htbp]
        \centering
        \lilypondfile{./potw-chords_intro.ly}
        \caption{Progressione dell'introduzione}
        \label{sheet:potw-chords_intro}
    \end{sheet}
    
    In queste poche battute ci sono svariati elementi che fanno pendere il centro tonale verso Sol maggiore: il primo accordo è un \writechord{G}, la durata degli accordi è tale da rendere il \writechord{C} quasi un accordo di passaggio, più simile a un abbellimento del \writechord{G} che non un accordo che porta l'armonia in qualche direzione. Infine, la cadenza plagale che si crea tra il \writechord{C} e il \writechord{G} ([ IV - I ]) è una cadenza molto blanda, e rafforza quindi la stabilità del \writechord{G} e l'ipotesi che possa essere il centro tonale del brano. Tuttavia, a causa dell'assenza di una dominante \writechord{D,7}, in questo stadio non è ancora possibile determinarlo.

    La progressione (Rg.\ref{sheet:potw1-chords_line1}) che abbiamo per i primi due versi  è simile alla progressione dell'introduzione, ma le prime due misure traslano il vamp di una quarta: al posto di \writechord{G} e \writechord{C} abbiamo rispettivamente \writechord{C} e \writechord{C,_,sus4}. La durata ritmica degli accordi rimane invariata, pertanto possiamo considerare \writechord{C,_,sus4} un abbellimento del precedente \writechord{C} per lo stesso motivo di cui sopra, nonostante l'uso del Fa naturale, non diatonico a Sol maggiore. Le successive due misure, invece, riprendono il vamp non traslato. 
    
    \begin{sheet}[htbp]
        \centering
        \lilypondfile{./potw1-chords_line1.ly}
        \caption{Progressione della sezione Linea 1 in \acrshort{pw1}}
        \label{sheet:potw1-chords_line1}
    \end{sheet}
    
    Interessante notare che in \acrshort{pw2} l'abbellimento sugli ultimi \meter{2}{8} viene sostituito con un \writechord{F} (Rg.\ref{sheet:potw2-chords_line1}); questa scelta non modifica l'analisi della sezione riportata, ed è presumibilmente motivata dalla volontà di aggiungere un elemento di varietà tra le due versioni del brano. Si può anche aggiungere che \writechord{F} è privo della dissonanza data dall'intervallo di seconda maggiore del \writechord{C,_,sus4}, e quindi è più consonante; cosi come il narratore  è più sereno e fiducioso, anche l'armonia diventa più consonante e stabile in \acrshort{pw2}.

    \begin{sheet}[htbp]
        \centering
        \lilypondfile{./potw2-chords_line1.ly}
        \caption{Progressione della sezione Linea 1 in \acrshort{pw2}}
        \label{sheet:potw2-chords_line1}
    \end{sheet}
    
    Astraendo la progressione dai dettagli, possiamo notare che la progressione  fino a questo punto riprende le prime 8 misure di un blues a 12 battute: abbiamo infatti quattro misure di \writechord{G} ([ I - I - I - I]), due misure di \writechord{C} e due misure di \writechord{G} ([ IV - IV - I - I]). Tuttavia, anziché proseguire con la canonica conclusione [ V - IV - I - I ], il gruppo [ IV - IV - I - I ] viene ripetuto e, come vedremo, la progressione prende tutt'altra direzione.

    
    Il fatto di iniziare una progressione armonica con un accordo di sottodominante (in questo caso \writechord{C}) le conferisce un carattere di leggero moto. Notiamo inoltre che la frase della voce recita solamente metà del verso (\gquotes{If you didn't care}, nel caso della prima ripetizione) e l'ultima nota della frase  cade sul battere del primo quarto della prima misura della progressione; abbiamo poi una seconda frase che recita la seconda metà del verso (\gquotes{what happens to me}). Il verso viene quindi lasiato a metà, sospeso, durante l'accordo di sottodominante, per poi essere completato durante l'accordo di tonica, giustificando quindi il ruolo dell'accordo di sottodominante come motore per questa progressione.
    
    \begin{sheet}[htbp]
        \centering
        \lilypondfile{./potw1-lyrics_line1.ly}
        \caption{Linea vocale in Linea 1 in \acrshort{pw1}}
        \label{sheet:potw1-lyrics_line1}
    \end{sheet}
    
    Fino a questo momento l'armonia del brano è rimasta stabile intorno al centro tonale di Sol maggiore; la situazione cambia nel verso successivo. Abbiamo infatti due misura di \writechord{A,7}. Contestualizzare questo accordo non è semplice: rispetto all tonica, si tratta di un [ \writechord{II,7} ] e contiene quindi un Do\sharp~ non diatonico a Sol maggiore e, a differenza del Fa naturale di cui abbiamo parlato prima, questo accordo viene tenuto per due misure, per cui non può essere inquadrato come un accordo di passaggio o un abbellimento. Di seguito, alcune potenziali giustificazioni per l'uso di questo accordo:
    \begin{itemize}
        \item \emph{dominante secondaria}: \writechord{A,7} potrebbe essere usato per preparare una modulazione verso Re maggiore che, tuttavia, non avviene;
        \item \emph{interscambio modale}: \writechord{A,7} potrebbe essere un prestito dalla parallela lidia, di cui fa parte, e suggerire quindi un allontanamento dall'armonia tonale in favore di una modale; tuttavia, non ci sono particolari elementi che suggeriscano questa ipotesi (come ad esempio un pedale di Sol), e la progressione di accordi che lo segue (Rg.\ref{sheet:potw1-chords_line2}) indebolisce ulteriormente questa tesi.
    \end{itemize}

    \begin{sheet}[htbp]
        \centering
        \lilypondfile{./potw1-chords_line2.ly}
        \caption{Progressione della sezione Linea 2 in \acrshort{pw1}}
        \label{sheet:potw1-chords_line2}
    \end{sheet}

    In seguito a queste due misure di \writechord{A,7}, infatti, abbiamo (Rg.\ref{sheet:potw1-chords_line2}) una progressione [ I - IV - V - I ] in Sol maggiore che, in quanto contiene la cadenza perfetta [ V - I ], rafforza sia la tesi del fatto che questo brano sia tonale e non modale, sia la tesi che il centro tonale sia Sol maggiore. La scelta del \writechord{A,7}, in quest'ottica, non sembra avere una motivazione chiara, quasi come se fosse casuale; tuttavia, è possibile che proprio questo straniamento sia la sensazione che Waters voleva trasmettere in un verso dove il narratore descrive un'umanità che, senza l'affetto reciproco, si sente persa e confusa, un'umanità in cui \gquotes{would zig-zag our way through the boredom and pain}\footnote{\gquotes{si andrebbe avanti a zig-zag tra noia e dolore}}.

    Vediamo quindi un uso dell'elemento armonico completamente al servizio dell'elemento lirico: allo straniamento suscitato da \writechord{A,7} segue subito un ritorno sul tracciato prestabiliti dettato dalla progressione più tonale che si possa scegliere, che ha il sapore di uno di quei pochi momenti di stabilità della vita umana in cui  \gquotes{occasionally glancing up through the rain}\footnote{\gquotes{alzando lo sguardo nella pioggia di tanto in tanto}}.

    Ma in una vita senza amori e affetti la serenita non può che essere effimera: subito dopo, infatti, la pusillanimitudine riprende il sopravvento negli umani che \gquotes{wondering which of the buggers to blame}\footnote{\gquotes{ponderano di quale accollo lamentarsi}}, e la tonalità di riferimento da maggiore diventa minore. Un \writechord{Am} funge da accordo pivot e sposta brevemente il centro tonale verso il La minore, relativa minore di Do maggiore, tonalità adiacente a Sol maggiore nel circolo delle quinte. Segue un breve scambio tra \writechord{Am} e \writechord{F} ([ Im ] e [ \flat VI ] di La minore): il rimpallo tra un accordo di area di tonica e uno di sottodominante ben rappresenta la frenesia interiore di chi sta cercando un capro espiatorio per il proprio malessere. Questa progressione si chiude con una misura di \writechord{C} che scende diatonicamente verso la tonica di \writechord{Am} (Rg.\ref{sheet:potw1-chords_line3}).

    \begin{sheet}[htbp]
        \centering
        \lilypondfile{./potw1-chords_line3.ly}
        \caption{Progressione della sezione Linea 3 in \acrshort{pw1}}
        \label{sheet:potw1-chords_line3}
    \end{sheet}

    A questo punto, come in ogni storia, abbiamo la realizzazione finale (Rg.\ref{sheet:potw1-chords_line4}). In questo caso l'accordo che la supporta è un \writechord{D}. Normalmente, un \writechord{D,7} sarebbe molto piu appropriato, in quanto sarebbe dominante secondaria di La minore e potrebbe quindi riportare l'armonia verso il centro tonale iniziale di Sol maggiore e, al contempo, confermare quest'ultima come tonalità del brano. In questo caso, però l'accordo è una triade semplice. Pur non possedendo la stessa tensione armonica della rispettiva quadriade, la triade in questo caso esercita comunque la funzione di dominante: la nota al canto è la sensibile Fa\sharp~ che risolve nel successivo \writechord{G} nella tonica Sol. 
    
    Se estendiamo il raggio di analisi possiamo anche considerare il \writechord{Am} che precede il \writechord{D}, che avrebbe quindi la funzione di accordo pivot e sopratonica secondaria, risultando quindi in un uso della cadenza composta tipica del jazz del [ IIm - V - I ]. Questa progressione è quindi un esempio di modulazione indiretta, in cui il \writechord{Am} funge da ponte tra le due tonalità.

    \begin{sheet}[htbp]
        \centering
        \lilypondfile{./potw1-chords_line4.ly}
        \caption{Progressione della sezione Finale.}
        \label{sheet:potw1-chords_line4}
    \end{sheet}
    
    La progressione dell'assolo riprende la maggior parte degli elementi armonici che abbiamo visto finora. In particolare, l'assolo inizia con due ripetizioni della progressione armonica della sezione Linea 1 (quindi [ IV - IV - I - I ]), leggermente alterata per seguire il metro che nell'assolo è \meter{6}{8}. Sono rimossi anche gli accordi di abbellimento. Il solo prosegue usando un vamp [ IIm - V ], che viene ripetuto tre volte, seguito da una ripresa della discesa diatonica (vedi Rg.\ref{sheet:potw1-rhythm_line23}). L'assolo termina con quattro misure che riprendono il vamp dell'introduzione, questa volta nel metro originale \meter{3+3+2}{8}.

    \begin{sheet}[htbp]
        \centering
        \lilypondfile{./potw-chords_solo.ly}
        \caption{Progressione dell'assolo.}
        \label{sheet:potw-chords_solo}
    \end{sheet}

\end{document}