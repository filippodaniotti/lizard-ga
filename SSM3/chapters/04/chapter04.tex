% arara: lualatex: { shell: true }
% arara: latexmk: { clean: partial }
\documentclass[class=book, crop=false, oneside, 12pt]{standalone}
\usepackage{standalone}
\usepackage{../../style}
\usepackage{../../glossary}

\let\isCompiledFromMain\undefined
\usepackage{../../music-tools}

\graphicspath{{./assets/images/}}
\ifdefined\isCompiledFromMain
\else
    \setluaoption{ly}{includepaths}{./assets/scores/}
\fi

\begin{document}

    \chapter{Sheep}

    \section{Introduzione}
    \label{sec:04-intro}

    Sheep è un brano lungo poco più di dieci minuti. 
    durata comparabile a quella delle altri tre canzoni core dell'album

    posizione nel disco
    
    
    come quelle, non è una vera e propria suite, nel senso che non tutti gli elementi della scrittura musicale cambiano, ma solo alcuni. es la tonalità non cambia (o comunque non sempre), nemmeno il bpm. in ultima analisi, pare più una canzone lunga che una suite.

    note di arrangiamento
    genesi e significato
    altre note



    \section{Struttura}
    \label{sec:04-struttura}

    Il brano si può dividere in \emph{tre} sezioni principali.
    La suddivisione del brano in sezioni si basa su criteri che individuano similitudini intra segmento e differenze extra segmento; in particolare, le caratteristiche principali tenute in considerazione sono dinamica, tonalità e modo, progressione di accordi, voci impiegate e trame melodiche.

    Per comodità consultativa, in questa analisi chiamiamo le tre sezioni rispettivamente \(\mathcal{A}\), \(\mathcal{B}\) e \(\mathcal{C}\). Il diagramma in Figura~\ref{fig:04-sections-timeline} mostra la posizione dei confini tra le sezioni individuate.

    \begin{figure}[htbp]
        \centering
        \subimport{assets/figures}{sections_timeline.tex}
        \label{fig:04-sections-timeline}
        \caption{Durata delle varie sezioni del brano.}
    \end{figure}

    \subsection*{Sezione \(\mathcal{A}\)}
    Sezione di apertura del brano, viene ripresa anche in un secondo momento. È caratterizzata da una dinamica contenuta, che va dal piano al mezzopiano. Dal punto di vista armonico la sezione è prevalentemente occupata da un pedale di D tenuto dal basso; sulla parte conclusiva, in funzione di una modulazione verso la sezione successiva, il pedale viene abbandonato in favore di un movimento verso la sottodominante Am. Lo spazio al di sopra del pedale armonico viene occupato in \(\mathcal{A}1\) dall'iconico assolo di piano elettrico, eseguito con un Rhodes Mk I, mentre in \(\mathcal{A}2\) da una trama armonico/melodica costituita da vari sintetizzatori e campioni di registrazioni ambientali.

    % Di seguito, descriviamo in dettaglio lo sviluppo ciascuna iterazione.
    
    \paragraph{\(\mathcal{A}1\)} 
    Il brano si apre con dei rumori ambientali campionati, nello specifico belati di pecora e cinghuettii di uccelli, cui subito  subentra l'assolo (minito 0:02). A 0:14 il basso entra in supporto al pedale armonico, con un lento fade in. A 1:20 il cambio di armonia chiama la chiusura dell'assolo. A 1:32 entra la batteria, poco prima dell'inizio della sezione \(\mathcal{B}\).

    \paragraph{\(\mathcal{A}2\)} 
    La ripresa avviene a minuto 5:32. A minuto 6:48 il il pedale di D viene abbandonato, mentre a 7:02 la batteria rientra.

    \subsection*{Sezione \(\mathcal{B}\)}

    
    \subsection*{Sezione \(\mathcal{C}\)}
    Sezione di chiusura del brano, di cui costituisce a tutti gli effetti una coda. Mantiene la dinamica della precedente sezione ed è caratterizzata da una tensione armonica molto intensa. Inizia circa a minuto 8:06 e si ripete in loop fino al fade out, il quale inizia circa a minuto 9:20. Dopo il minuto 10:00 nessuno strumento è più udibile e restano solo i belati delle pecore, in ripresa dell'inizio del brano.
    
    \section{Tecnica e arrangiamento}
    \label{sec:03-arrangement}

    \subsection{Assolo di Rhodes}

    \begin{sheet}[htbp]
        \centering
        \lilypondfile{./sheep-epiano_solo.ly}
        \caption{Pattern ritmico dell'introduzione.}
        \label{sheet:sheep-epiano_solo}
    \end{sheet}

    \section{Analisi armonica}
    \label{sec:03-harmony}

    La sezione \(\mathcal{A}\) è caratterizzata nella sua quasi totale interezza da un pedale di D. In \(\mathcal{A}1\), L'assolo di Rhodes fa ampio uso del Si naturale, dando alla sezione una forte impronta modale dorica. 
    \todo[inline]{Interessante la modulazione indiretta verso Em.} 
    In \(\mathcal{A}2\), invece, la trama cambia profondamente e al di sopra del pedale viene costruita una fortissima tensione attraverso l'ampio uso dell'accordo di settima diminuita, che viene impiegato dai vari synth e organi, principalmente in forma di arpeggio dei primi e cluster di note dai secondi. \todo[inline]{Interessante che all'inizio solo due note, quindi diatoniche e tutto normale, e poi solo }

    La sezione \(\mathcal{B}\) inizia con una semplice progressione in forma [ I - IV ], quindi Em e C. Gli accordi hanno un respiro molto ampio e coprono ciascuno la durata di \(4\) misure, rivelandosi pienamente in funzione dell'elemento lirico. 
    Modulazione a F\#, con alternanza della IIIm cioè A

    Infine, la sezione \(\mathcal{C}\) appare inizialmente modale. Le trame armoniche sono statiche sul Mi mentre la chitarra esegue un ostinato costruito su una discesa diatonica sul Mi misolidio; questo schema viene ripetuto molte volte, restituendo la ragionevole certezza che l'intera sezione sia un pedale di Mi. Tuttavia, dopo n ripetizioni l'armonia si sposta su un A, spiazzando l'ascoltatore e forzando a ricontestualizzare le trame armoniche precedenti: questo passaggio ha un forte elemento risolutivo, in opposizione alla tensione dominante del precedente E7 e dovuta al tritono. Questa sezione appare quindi tonale, anzi: può essere vista come una cadenza perfetta ([ V7 - I ]), dove la risoluzione verso il I arriva dopo un periodo di tensione lungo e straziante.
    

\end{document}