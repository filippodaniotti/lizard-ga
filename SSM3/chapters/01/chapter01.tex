\documentclass[class=book, crop=false, oneside, 12pt]{standalone}
\usepackage{standalone}
\usepackage{../../style}
% \usepackage{../../music-tools}


\provideboolean{isCompiledFromMain}
\setboolean{isCompiledFromMain}{false}

\begin{document}
    \chapter{I Pink Floyd}

    \section{Biografia Essenziale}
    \subsection{Nascita e primi anni}
    Il primo nucleo della band che successivamente sarebbe diventata nota con il nome di Pink Floyd si può individuare nei \emph{Sigma 6}, gruppo nato dall'incontro tra Nick Mason e Roger Waters nei corridori della facoltà di architettura della \emph{University of Westminster} a Londra nel 1963. Successivamente noti come \emph{The Tea Seat} e dediti principalmente a cover di brani R\&B, il gruppo era composto da Nick Mason alla batteria, Roger Waters alla chitarra solista, Richard Wright alle tastiere oppure, a volte, alla chitarra ritmica; inizialmente vi facevano parte anche Keith Noble e Clive Metcalfe, che però presto avrebbero lasciato il complesso. Come rimpiazzo, Waters imbracciò il basso e il gruppo ingaggiò Bob Klose alla chitarra solista e Chris Dennis come voce solista; a questi si aggiunse inoltre il chitarrista Syd Barrett, vecchio amico d'infanzia di Waters.

    È con questa formazione che il gruppo riesce a fare le prime sedute in studio nel dicembre 1964, ospiti nello studio di registrazione di un amico di Wright; poco tempo dopo, nel 1965, divennero resident band per un locale di nome \emph{The Countdown Club}, riuscendo a suonare tre set da 90 minuti al giorno. Nello stesso anno, Dennis avrebbe lasciato la band poiché coscritto nella \emph{Royal Air Force}; il suo abbandono sarà presto seguito da quello di Klose, mosso da pressioni da parte dei genitori rispetto alla carriera accademica. Barrett assunse quindi il duplice ruolo di chitarrista e cantante solista, cristallizando quindi la prima formazione storica della band. A testimonianza del ruolo di leadership ricoperto da Barrett, è in questo periodo che abbiamo il cambio di nome definitivo: fu infatti Barrett a coniare \emph{Pink Floyd}, nome costruito accostando i nomi di due dei suoi artisti blues americani preferiti, \emph{Pink Anderson} e \emph{Floyd Council}.

    \subsection{L'era Barrett}
    In questo periodo il gruppo acquisisce progressivamente notorietà attraverso una proficua attività live e, sotto la leadership di Barrett, introduce al sostrato R\&B elementi rock e blues. Il loro stile venne presto identificato sotto l'ombrello del \emph{rock psichedelico}, anche a seguito dell'uso di sostanze allucinogene fatto dalla band, in particolare da Barrett. In questo periodo ilgruppo fu notato da \emph{Peter Jenner} e \emph{Andrew King}, i quali restarono colpiti dall'esecuzione e dallo stile della band e si offrirono nel ruolo di manager. Questa collaborazione si tradusse in un salto di qualità in frequenza e magnitudine negli eventi live dei Pink Floyd, al netto delle perplessità mosse da alcuni esercenti sullo stile; alle cover R\&B il gruppo aveva progressivamente affiancato i lavori sperimentali di Barrett, i quali non erano sempre ben ricevuti: ad esempio, a seguito di un concerto presso il \emph{Catholic youth club} il titolare si rifiutò di pagarli perché, a detta sua, la performance "non era musica". 
    
    Nonostante questo, alla fine del 1966 il gruppo aveva costruito una discreta fanbase e iniziato a suscitare l'interesse delle etichette discografiche, in particolare di \emph{EMI Columbia}, che garantì al gruppo \pounds 5000 per la produzione e successiva pubblicazione del primo singolo, \emph{Arnold Layne}, nel marzo 1967. A dispetto del testo quantomeno bizzarro e tabù (il brano parla infatti di un uomo che ruba e indossa abiti e biancheria femminili), il brano ebbe un discreto passaggio radiofonico, tanto da giustificare la produzione di un video promozionale. A giugno uscì un secondo singolo, \emph{See  Emily Playin'}, che riscosse un successo ancora maggiore e solidificò le basi per la produzione del primo LP della band, \emph{The Piper at the Gates of Dawn}, nell'agosto dello stesso anno; il disco fu un successo e fu pubblicato sia sul mercato europeo, sia in quello statunitense. 
    
    In questo periodo l'uso di allucinogeni da parte di Barrett si intensificò e il resto del gruppo iniziò a osservare spiacevoli episodi di dissociazione da parte già durante la produzione del disco; questi si intensificarono durante il tour promozionale, e costrinsero il gruppo ad eliminare le date negli Stati Uniti a causa del deterioramento della stabilità mentale di Barrett, che andava spesso incontro a esaurimenti nervosi e crisi di panico. Inoltre, nei concerti effettuati, Barrett era spesso assente, dava spesso le spalle al pubblico e, spesso, affossava l'intera performance rifiutandosi di cantare e suonando una sola nota o un solo accordo per l'intera durata dell'evento. Per questi motivi il gruppo introdusse nella band \emph{David Gilmour}, un vecchio amico di Barrett. 
    
    Inizialmente Gilmour avrebbe dovuto suonare live con la band qualora Barrett fosse indisposto, ma ben presto nel condizioni di salute di Barrett peggiorarono al punto che era evidente che soffrisse di depressione e forse schizofrenia. Aveva smesso di partecipare alle sedute di registrazione, non parlava con nessuno degli altri membri della band e il suo contributo alla scrittura dei pezzi si ridusse al punto che, nel secondo LP \emph{A Saucerful of Secrets} (giugno 1968), un solo pezzo di Barrett era presente: \emph{Jugband Blues}, un brano alquanto particolare che esprime adeguatamente il misterioso e contorto stato interiore di Barrett, con uno spiraglio di consapevolezza che, a causa dello stesso, la sua esperienza nell'industria musicale stava per volgere al termine.

    Barrett infatto fu presto estromesso dalla band e si ritirò a vita privata, rifuggendo ogni forma di apparizione pubblica e vivendo il resto della sua vita nell'anonimato. Questa scelta con ogni probabilità contribuì a cementificare lo status di mistero e dannazione che avvolge la figura di Barrett, sia di fronte al grande pubblico, ma anche nei confronti degli stessi Pink Floyd

    \subsection{L'era Waters}
    Al netto degli inconvenienti causati da Barrett, la produzione e il rilascio di \emph{A Saucerful of Secrets} ottennero buoni risultati e il disco riuscì a posizionarsi nella nona posizione nelle \emph{UK Charts}. Questo successo promosse un tour transcontinentale in Europa e negli Stati Uniti e la produzione dei due dischi successivi: \emph{More} (giugno 1969), colonna sonora del film omonimo, e \emph{Ummagumma} (novembre 1969). Quest'ultimo segnò un notevole stacco dai precedenti lavori, in termini di ambizione e sperimentazione: si trattava infatti di un doppio album, la cui prima metà era costituita da delle registrazioni live, mentre la seconda metà da materiale inedito e dal carattere fortemente sperimentale; nonostante la tiepida accoglienza della critica, il gruppo successivamente avrebbe espresso un certo disprezzo nei confronti del disco. Anche il successivo LP  \emph{Atom Heart Mother} (ottobre 1970) ebbe un destino simile: fu infatti il primo lavoro nel quale la band collaborò estesamente con altri musicisti (il compositore \emph{Ron Geesin} e il direttore d'orchestra \emph{John Alldis}), nonché il primo disco del gruppo a raggiungere la prima posizione delle \emph{UK Charts}, ma Gilmour e Waters si sarebbero sempre mostrati critici del lavoro.

    Il successivo LP, \emph{Meddle} (ottobre 1971), segna un passo importante nella carriera dei Pink Floyd per due motivi: in primo luogo, è il primo disco in cui l'elemento sperimentale e \emph{progressive} diventa trainante e preponderante nel suono e nel marketing della band, perdendo ormai del tutto l'eredità psichedelica dell'era Barrett; in secondo luogo, è qui che la leadership di Waters in termini di testi e scrittura inizia a consolidarsi, nonostante a questo punto il contributo degli altri membri sia ancora importante, in particolare quello di Gilmour. Il successivo disco \emph{Obscured by Clouds} (1972), conferma queste tendenze.

    Obscured by Clouds fu però quasi subito oscurato dal suo successore. La band tornò agli \emph{Abbey Roads Studios} nel maggio del 1972 e iniziò, con l'assistenza dell'ingegnere \emph{Alan Parsons}, a lavorare a delle demo che Waters aveva realizzato in un piccolo studio allestito nella sua capanna degli attrezzi; tutti i testi furono scritti dallo stesso Waters, e componevano un concept album i cui argomenti includevano avidità, conflitto, morte e salute mentale. Le 10 tracce del disco erano pensate per essere riprodotte senza soluzione di continuità, a riconferma della concezione organica tipica dei concept album del prog di quegli anni. Inoltre, il gruppo fece un uso ampio e creativo di sintetizzatori, campionatori, strumentazione ed effettistica, portando al grande pubblico dei suoni del tutto nuovi. Per tutti questi motivi \emph{The Dark Side of the Moon}, rilasciato nel marzo 1973, può essere considerato il primo disco compiutamente progressive dei Pink Floyd; nonostante questo, fu fin da subito il loro maggior successo commerciale, raggiungendo la prima posizione della \emph{Billboard 200 albums} e rimanendo nella classifica per 736 settimane; vendette 46 milioni di copie, risultando \emph{de facto} il terzo disco più venduto nella storia dell'industria discografica.

    Il gruppo mantenne gran parte degli elementi progressive anche nel successivo \emph{Wish You Were Here} (1975), nel quale i testi di Waters approfondivano il tema dell'alienazione, sia dal punto di vista delle relazioni, sia dal punto di vista dell'industria discografica. Waters trovò grande ispirazione in questo senso nella figura di Barrett e nel suo decadimento. Un episodio notevole avviene il 5 giugno 1975, quando Barrett si presentò in studio senza preavviso. A quel punto, Barrett era diventato calvo, con le sopracciglia rasate e leggermente sovrappeso, tanto che inizialmente non lo riconobbero e, dopo aver realizzato la sua identità, alcuni della squadra erano in lacrime. Il gruppo a quel punto suonò \emph{Shine on You Crazy Diamond}, ma pare che Barrett non fosse sufficientemente lucido da capire che il pezzo era dedicato a lui.

    Il successivo \emph{Animals} (1977) da un lato continua il ramo dei concept album, questa volta proponendo una critica alla condizione socio-politica della società inglese degli anni '70 traendo profonda ispirazione dal romanzo di Orwell \emph{Animals Farm}; dall'altro, segna la formazione di una prima crepa nella formazione. Waters stava infatti assumendo un controllo sempre più profondo sulla scrittura e la direzione musicale dei Pink Floyd, e questo causava delle discussioni di carattere sia personale, sia pecuniario. 
    
    Il tour promozionale di Animals fu inoltre segnato da alcuni incidenti con alcuni spettatori degli eventi live: ad esempio, a seguito di un concerto a Montreal, Waters sputò in viso a un fan che lo aveva raggiunto per dirsi insoddisfatto della performance di quella sera. Questi episodi ispirarono Waters nella scrittura del successivo \emph{The Wall} (1979), un doppio concept album su alienazione, abbandono e solitudine vista attraverso gli occhi di \emph{Pink}, una fittizia rockstar depressa in preda a delle riflessioni profonde sulla propria carriera. Rilasciato in fretta a causa di incombenti difficoltà economiche, il disco si rivelò fortunatamente un successo, vendendo più di 30 milioni di copie; vennero estratti tre singoli (\emph{Another Brick in the Wall, Part 2}, \emph{Run Like Hell} e \emph{Comfortably Numb}) e tutti e tre diventarono hit radiofoniche. 
    
    La produzione di The Wall vide anche l'estromissione dalla band di Wright, il quale stava attraversando alcuni problemi personali che gli impedivano di dare qualsivoglia tipo di contributo alla produzione del disco; nonostante questo, partecipò lo stesso al tour in qualità di session man stipendiato. Conseguentemente, Wright non partecipò nemmeno alla produzione del successivo \emph{The Final Cut} (1983). Il disco ricevette un'accoglienza tiepida e fu il disco meno venduto dai tempi di Obsucred by Clouds; questo, assieme a tanti altri motivi di carattere personale e musicale, alimentò il crescente conflitto tra Waters e Gilmour. Queste tensioni sfociarono nell'abbandono della band da parte di Waters, convinto che il gruppo non avesse più alcun futuro. Gilmour invece, una volta vinta la diatriba legale con Waters per il controllo del nome, decise assieme a Mason di reintrodurre Wright e portare avanti l'esperienza dei Pink Floyd.

    \subsection{L'era Gilmour}
    Assieme al produttore Bob Ezrin, che già aveva collaborato con il gruppo per la produzione di The Wall, la band mise assieme del materiale che confluì in \emph{A Momentary Lapse of Reason} (1987), disco che ottenne un discreto successo e dal quale furono estratti tre singoli che ebbero un considerevole passaggio radiofonico (tra i quali \emph{Learning to Fly}). In seguito al tour promozionale, le attività a nome dei Pink Floyd vennero messe in pausa  e i tre membri del gruppo perseguirono ciascuno delle proprie attività da solista. Nel 1993 i tre si trovarono in studio assieme a Bob Ezrin per registrare nuovo materiale, in gran parte preparato da Gilmour e Wright. Il risultato fu \emph{The Division Bell} (1994), seguito anch'esso da un imponente tour promozionale. In seguito, le attività della band si fermarono per circa vent'anni.

    Nel 1997 il gruppo fu invitato a Los Angeles per essere introdotto nella \emph{Rock And Roll Hall of Fame}, ma solamente Gilmour, Wright e Mason si presentarono; Barrett non rispose alla chiamata, e Waters rifiutò ogni sorta di coinvolgimento. Nove anni dopo, in occasione del \emph{Live Aid} a \emph{Hyde Park}, avvenne il tanto agognato ricongungimento di Waters con il resto della band; il concerto fu un successo enorme, ma non fu seguito da alcun tipo di ripresa delle attività in studio o live. Tuttavia, nel 2007 il gruppo organizzò un grande concerto in memoria di Barrett, morto l'anno prima di cancro al pancreas.

    L'anno successivo, nel 2008, Wright morì di un tumore ai polmoni. Questo evento fu motivatore per ill ritorno in studio di Gilmour e Mason, i quali recuperarono parte del materiale scartato durante la produzione di The Division Bell e lo completarono, rilasciando quindi, nel 2014, \emph{The Endless River}, quindicesimo e ultimo album dei Pink Floyd.
    

    \section{Influenze}
    \section{Evoluzione musicale}
    \section{Lascito}
\end{document}