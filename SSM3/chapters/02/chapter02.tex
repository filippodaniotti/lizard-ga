\documentclass[class=book, crop=false, oneside, 12pt]{standalone}
\usepackage{standalone}
\usepackage{../../style}
\usepackage{../../glossary}
% \usepackage{../../music-tools}
\graphicspath{{./assets/images/}}

\newcommand*{\wywh}{\emph{Wish You Were Here}}


% \provideboolean{isCompiledFromMain}
% \setboolean{isCompiledFromMain}{false}


% arara: lualatex
% arara: biber
% arara: latexmk: { clean: partial }
\begin{document}
    \chapter{Animals}
    \section{Produzione}
    \section{Distribuzione}
    \section{Remaster e riedizioni}
    \section{Profilo artistico e musicale}

    \subsection{Concept e tematiche}
    \label{subsec:animals-concept}
    Ispirato al romanzo Animal Farm di George Orwell, l'idea alla base di \acrshort{anm} è la rappresentazione delle classi sociali come  animali: 
    \begin{itemize}
        \item i  \emph{cani}, aggressivi, sono i rappresentanti della legge;
        \item i  \emph{maiali}, dispotici e spietati, rappresentano i politici;
        \item le \emph{pecore} non sono altro che la mandria insana e cieca che tiene in vita il potere.
    \end{itemize}
    
    L'album, a differenza dell'opera di Orwell, metafora del regime stalinista, è una critica feroce al sistema capitalista nell'Inghilterra degli anni Settanta, e differisce nella rivolta finale delle pecore nei confronti dei cani.

    \subsection{Direzione compositiva}
    \label{subsec:animals-songwriting}
    Sotto il profilo più strettamente musicale, Animals è un album che presenta elementi sia di continuità, sia di rottura con la direzione tracciata dai lavori precedenti. 
    
    Da un lato, infatti, il disco mantiene inalterate la struttura organica e la sinergia tra l'elemento musicale e l'elemento lirico. Delle cinque tracce che costituiscono la tracklist, infatti, la prima e l'ultima fungono  da cornice narrativa, e le tre tracce centrali sono dei brani più lunghi che costituiscono il nucleo centrale dell'album. In questo senso, si può dire che Animals ricalca la stessa struttura adottata dal precedente \acrlong{wywh}; tuttavia, mentre in \acrshort{wywh} il prologo ed epilogo erano costituiti da una lunga \emph{suite} divisa in due parti, in \acrshort{anm} sono due brani molto brevi e semplici in termini di arrangiamento ed esecuzione, e che condividono la medesima struttura armonica e melodica, differendo, di fatto, solo per il testo. 

    \acrshort{anm} mantiene anche molti degli elementi caratteristici del progressive (in salsa) Floyd: i tre brani centrali sono \emph{suite} lunghe e articolate, la ricerca della complessità viene mantenuta tanto nelle scelte armoniche, quanto in quelle melodiche (notevole in questo senso l'uso di una frase armonizzata sulla scala esatonale nel solo di Dogs), gli arrangiamenti sono ricercati e ampio spazio viene riservato alle parti strumentali.

    Dall'altro lato, però, \acrshort{anm} si discosta dai lavori precedenti per alcuni aspetti. In primo luogo, l'album è caratterizzato da un'atmosfera più cupa e tetra rispetto ai precedenti che, in accordo con le tematiche trattate, si riflette nella scelta dei suoni e degli arrangiamenti. In secondo luogo, l'album è caratterizzato da un abbandono delle sonorità affini al \emph{jazz}, molto presenti fino a quel momento e manifestatesi più compiutamente in \acrshort{dsom}, in favore di sonorità \emph{rock} più tradizionali. Questo si declina  in un uso più ampio e marcato della chitarra elettrica e, soprattutto, in una preminenza di tastiere elettromeccaniche (Hammond e Rhodes) a sfavore dei sintetizzatori, che nel precedente \acrshort{wywh} erano quasi protagonisti.


\end{document}