\documentclass[class=book, crop=false, oneside, 12pt]{standalone}
\usepackage{standalone}
\usepackage{../../style}
\graphicspath{{./assets/images/}}

\begin{document}
\chapter*{Prefazione}
\addcontentsline{toc}{chapter}{Prefazione}

\begin{itemize}
    \item la notazione accordi è quella usata in armonia musicale moderna p 47 tranne in quei casi in cui mi sono arreso nella battaglia contro lilypond, ossia:
    \begin{itemize}
        \item \writechord{Gdim} e \writechord{G(b5)} sono la stessa cosa
        \item G\(^\Delta\) è \writechord{Gmaj7}
    \end{itemize}
    \item regole per le reference: Item.x.x
    \item regole per il noem di sezioni: ripetizioni, iterazioni/occorrenze
\end{itemize}

\end{document}