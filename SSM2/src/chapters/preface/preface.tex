\documentclass[class=book, crop=false, oneside, 12pt]{standalone}
\usepackage{standalone}
\usepackage{../../style}
\graphicspath{{./assets/images/}}

\begin{document}
\chapter*{Prefazione}
\addcontentsline{toc}{chapter}{Prefazione}
Questo elaborato costituisce la guida all'ascolto per l'esame di Pianoforte Moderno di livello SSM2. Di seguito, alcune note utili a una più agevole fruizione del testo.

\begin{itemize}
    \item La notazione scelta per le sigle di accordi è quella proposta nel testo \emph{Armonia musicale moderna}, p. 47, a esclusione dei seguenti casi:
    \begin{itemize}
        \item le triadi diminuite possono essere notate equivalentemente come \writechord{Gdim} e \writechord{G(b5)};
        \item gli accordi estesi che contengono una settima maggiore possono essere notati equivalentemente come G\(^\Delta\) è \writechord{Gmaj7}.
    \end{itemize}
    \item La notazione di batteria utilizzata è quella proposta dalla reference del software Lilypond\cite{res:lily-drum-chart}.
    \item Lo schema usato per i riferimenti interni al testo è il seguente: Item.X.X (ad esempio Sec.2.3 indica la sezione numero 3 del capitolo 2; Sp.3.11 indica lo spartito numero 11 del capitolo 3).
    \item Nella sezione dedicata alla struttura vengono assegnati degli identificatori univoci alle sezioni (\(A\), \(B\), \(special\)); tuttavia, il loro utilizzo non è rigido e, di quando in quando, vengono impiegati nomi più familiari come "strofa" o "ritornello"; tuttavia, il contesto dovrebbe rendere sempre evidente quale sezione venga indicata di volta in volta, ma si consiglia comunque di prestare attenzione.
\end{itemize}

\end{document}