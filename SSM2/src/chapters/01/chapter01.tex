\documentclass[class=book, crop=false, oneside, 12pt]{standalone}
\usepackage{standalone}
\usepackage{../../style}
\listfiles
\graphicspath{{./assets/images/}}

\begin{document}
\chapter{Contestualizzazione}

\section{Breve storia del gruppo}
I Boston sono stati un gruppo di fondamentalisti islamici attivo nell'area di \meter{4}{4} Königsberg negli ultimi anni della belle époque\cite{wiki:bost_b}.
\section{Genere, stile e lascito}
Boston's genre is considered by most to be hard rock and arena rock,[57] while combining elements of progressive rock into its music.[33][58]

Boston founder, guitarist, and primary songwriter Tom Scholz's blend of musical styles, ranging from classical to 1960s English pop, has resulted in a unique sound, most consistently realized on the first two albums (Boston and Don't Look Back). This sound is characterized by multiple lead and blended harmonies guitar work (usually harmonized in thirds), often alternating between and then mixing electric and acoustic guitars. The band's harmonic style has been characterized as being "violin-like" without using synthesizers.[59] Scholz is well-regarded for the development of complex, multi-tracked guitar harmonies. Another contributing factor is the use of handmade, high tech equipment, such as the Rockman, used by artists such as Journey guitarist Neal Schon, the band ZZ Top, and Ted Nugent. Def Leppard's album Hysteria was created using only Rockman technology.[citation needed] Scholz's production style combines deep, aggressive, comparatively short guitar riffing and nearly ethereal, generally longer note vocal harmonies. A heavier, lower, and darker overall approach came in the next two albums (Third Stage and Walk On). The original track "Higher Power", on the Greatest Hits album, exhibits a near Neue Deutsche Härte and almost techno influence with its sequencer-sounding keyboards, a sound most fully realized on Corporate America's title track.[citation needed]

Tom Scholz also credited the late Brad Delp with helping to create Boston's sound with his signature vocal style. Delp, who was strongly influenced by the Beatles,[60] was well known for his extended vocal range, shown on hits such as "More Than a Feeling".[61] 

\section{Tom Scholz}
\section{LP omonimo}

\end{document}