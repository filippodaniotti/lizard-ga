\documentclass[class=book, crop=false, oneside, 12pt]{standalone}
\usepackage{standalone}
\usepackage{../../style}
\listfiles
\graphicspath{{./assets/images/}}

\begin{document}
\chapter{Contestualizzazione}
I Boston sono un gruppo rock americano fondato nell'omonima città del Massachussets nel 1975. A fronte di numerosi cambi di formazione, la costante presenza del leader polistrumentista Tom Scholz e del cantante Brad Delp (fino alla morte nel 2007) hanno permesso alla formazione di mantenere identità e stile caratteristico per oltre quattro decadi di attività piuttosto discontinua. 

Il complesso ha all'attivito sei album in studio e oltre 75 milioni di dischi venduti, 25 dei quali relativi al solo primo disco omonimo, che risulta pertanto uno dei dischi di debutto più venduti della storia della musica hard rock. 

\section{Tom Scholz e i Boston}
La storia e la direzione musicale dei Boston sono intimamente legati alla figura di Donald Thomas Scholz, fondatore nonché principale compositore del gruppo.

Scholz colloca l'inizio del percorso musicale nel 1969 quando, mentre frequentava il MIT, ha scritto una prima versione di un brano musicale che sarebbe stato successivamente noto come Foreplay\cite{wiki:bost_b}. Sempre all'interno dell'ambiente del MIT Scholz entrò in un gruppo di nome Freehold, all'interno del quale conobbe il chitarrista Barry Godreau e il batterista Jim Masdea. È con quest'ultimo che Scholz collaborò più intensamente negli anni a venire.

Dopo aver conseguito il Master's Degree in ingegneria meccanica, Scholz trovò lavoro presso il dipartimento di product design presso Polaroid; contemporaneamente, continuava a suonare come tastierista in numerose formazioni dell'area di Boston, continuando a collaborare con Masdea e Godreau e arrivando a fare la conoscenza del cantante Brad Delp. 

Parallelamente a questo, sviluppò assieme a Masdea un acceso interesse per la registrazione e la produzione audio; l'interessere e la curiosità dei due musicisti era tale che allestirono una sorta di studio di registrazione amatoriale nel seminterrato di Masdea, all'interno del quale iniziarono a registrare delle demo. Masdea si occupava delle tracce di batteria, mentre Scholz di tutto il resto. Tutto l'equipaggiamento che il duo utilizzava per registrare era autocostruito da Scholz, che mise a frutto gli anni di studio al MIT e l'esperienza nel dipartimento di Product Design a Polaroid per costruire un rudimentale registratore a nastri. 

Il perfezionismo e la morbosa ricerca dei suoni particolari che aveva in mente portarono Scholz a non limitarsi alla costruzione della strumentazione tecnica necessaria a registrare, ma progettò anche gran parte dell'equipaggiamento per la chiatarra, dagli amplificatori ai pedali effetto. In un'intervista ha dichiarato che per lui la ricerca dei suoni e la composizione dei brani sono due attività che procedono di pari passo, e proprio grazie al suo essere sia un ingegnere sia un musicista può eliminare le difficoltà di comunicazione che solitamente sussistono tra queste due figure e creare esattamente i suoni che gli servono per i brani che ha in mente\cite{yt:scholz_sound}.

Nel biennio 1973-1974 Scholz ingaggiò Brad Delp per registrare la voce dei brani fino ad allora registrati e ultimò un disco demo contente More Than a Feeling, Peace of Mind, Rock \& Roll Band, Hitch a Ride e Don't Look Back, che spedì a tutte le maggiori etichette discografiche del periodo. Tutte quante rifiutarono la demo e Scholz, che ormai aveva quasi 28, considerava fondamentamentalmente conclusa la sua esperienza musicale.

Per una serie di fortunate coincidenze\cite{wiki:bost_a}, la demo di Scholz catturò l'attenzione di Paul Ahern e Charles McKenzie\cite{wiki:bost_b}, che proposero a Scholz un contratto con Epic Records. In quel periodo Masdea perse interesse nel progetto e lasciò la formazione poco prima che Scholz e Delp firmassero il contratto. La formazione venne dunque completata con Braay Godreau alla chitarra solista, Fran Sheehan al basso e Sib Hashian alla batteria.

Ci fu un breve screzio tra il produttore Scholz e il produttore John Boylan: questi avrebbe preferito infatti che tutte le tracce del disco fossero ri-registrate in uno studio professionale di Epic a Los Angeles, ma Scholz era assolutamente contrario all'idea di abbandonare la sua strumentazione e l'ambiente del seminterrato in cui aveva registrato le demo fino a quel momento. Raggiunsero quindi un compromesso: Scholz avrebbe ri-registrato le tracce di sua competenza (ovvero tuttei gli organi, i clavinet e la maggior parte delle chitarre e bassi) nel suo seminterrato, mentre il resto del gruppo avrebbe registrato nello studio Epic. A fine registrazione, le tracce sarebbero state quindi mandate a Boylan per procedere al mix e al mastering.

L'album venne rilasciato infine il 25 agosto 1976 e fu subito un incredibile successo, arrivando a vendere 17 milioni di copie nei soli Stati Uniti e toccando la posizione numero 3 nella Billboard 200 e restando in classifica per 132 settimane\cite{wiki:bost_b}. Vennero rilasciati tre singoli: More Than a Feeling, Long Time e Peace of Mind.

Il Successivo disco \emph{Don't Look Back} uscì nel 1978. Nonostante un intervello di due anni tra LP fosse considerato piuttosto ampio a quel tempo, Scholz accusò le pressioni della Epic e si dichiarò insoddisfatto del risultato. Il disco ebbe comunque un discreto successo, riuscindo a vendere oltre 7 milioni di copie.

A seguito di questo l'attività del gruppo divenne molto più sparsa e sporadica. Dal 1979 fino a oltre la metà degli anni '80 i membri del gruppo si dedicarono a progetti indivuali. In particolare, Scholz si dedicò alla progettazione e vendita di strumentazione per chitarra attraverso la sua compagnia \emph{Scholz Research \& Development, Inc.}, fondata nel 1980\cite{wiki:scholz_inc}; di grande successo fu in particolare la linea di amplificatori e processori di segnale Rockman, che registro buoni dati di vendita per tutti gli anni '80. La compagnia sopravvisse fino al 1995, quando Scholz stesso la chiuse e vendette la linea Rockman alla Dunlop Manufacturing, Inc\cite{wiki:scholz_inc}. 

Nonostante gli album rilasciati successivamente a questo periodo non riuscirono mai a replicare il successo prorompente ottenuto nei primi anni dalla band, Scholz mantenne piuttosto costantemente l'attività musicale. In questo modo abbiamo il rilascio di \emph{Third Stage} nel 1986, \emph{Walk On} nel 1994, un \emph{Greatest Hits} nel 1997 e \emph{Corporate America} nel 2002.

Il gruppo ricevette una profonda scossa nel 2007, quando il cantante Brad Delp si suicidò nella propria abitazione tramite l'inalazione di monossido di carbonio. Il 17 agosto dello stesso anno venne organizzato un concerto tributo a lui dedicato alla Bank of America Pavilon a Boston.

A seguito di questo, il gruppo proseguì una intermittente attività di touring. Nel 2013 venne rilasciato il piu recente disco a nome Boston, \emph{Live, Love \& Hope}, che ancora conteneva delle tracce vocali registrate da Delp. Nel 2017 Scholz ha annunciato un nuovo album in lavorazione, senza azzardare possibili date di uscita.

\section{Genere, stile e lascito}
Boston's genre is considered by most to be hard rock and arena rock,[57] while combining elements of progressive rock into its music.[33][58]

Boston founder, guitarist, and primary songwriter Tom Scholz's blend of musical styles, ranging from classical to 1960s English pop, has resulted in a unique sound, most consistently realized on the first two albums (Boston and Don't Look Back). This sound is characterized by multiple lead and blended harmonies guitar work (usually harmonized in thirds), often alternating between and then mixing electric and acoustic guitars. The band's harmonic style has been characterized as being "violin-like" without using synthesizers.[59] Scholz is well-regarded for the development of complex, multi-tracked guitar harmonies. Another contributing factor is the use of handmade, high tech equipment, such as the Rockman, used by artists such as Journey guitarist Neal Schon, the band ZZ Top, and Ted Nugent. Def Leppard's album Hysteria was created using only Rockman technology.[citation needed] Scholz's production style combines deep, aggressive, comparatively short guitar riffing and nearly ethereal, generally longer note vocal harmonies. A heavier, lower, and darker overall approach came in the next two albums (Third Stage and Walk On). The original track "Higher Power", on the Greatest Hits album, exhibits a near Neue Deutsche Härte and almost techno influence with its sequencer-sounding keyboards, a sound most fully realized on Corporate America's title track.[citation needed]

Tom Scholz also credited the late Brad Delp with helping to create Boston's sound with his signature vocal style. Delp, who was strongly influenced by the Beatles,[60] was well known for his extended vocal range, shown on hits such as "More Than a Feeling".[61] 

% \begin{itemize}
    % \item  father played rtumpet in a band. told himto stay away from music as a job, boston went good, he makes money by building and selling equipment for musicians. father be like wtf
    % \item ha scoperto i pedali e tipo wow; il suo preferito era l'yperspace pedal (così lo chiamava lui)
    % \item lo ha costruito lui ecco, ne esistono solo due
    % \item sustain infinito
    % \item la roba figa di essere both engineer and musician is that you eliminate all the communication and this is good since it's difficult to get a sound idea across by words
    % \item lui si faceva i pedali per i suoi che voleva lui, era stupito che altri musicisti gli scrivessero per chiedergli dei suoi sounding
    % \item jeff beck gli ha scritto per un rockman headphone amp
    % \item lui ha una composizione molto sound-oriented: ha un'idea ma non riesce a sentirla nel complesso finché non mette a posto del tutto i suoni, e se non esistono li inventa
    % \item lui ha una close relation coi suoni e la strumentazione
    % \item Lui ha registrato boston tutto homemade, non perché fosse sicuro ma perché era così che andava fatto; gli pareva che fosse lúnico modo per farlo suonare right,ed è l'unico modo che ha per registrare musica
    % \item era molto insicuro sia della sua tecnica che della sua musica
    % \item 
% \end{itemize}

\section{LP omonimo}

\end{document}