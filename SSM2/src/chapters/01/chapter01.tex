\documentclass[class=book, crop=false, oneside, 12pt]{standalone}
\usepackage{standalone}
\usepackage{../../style}
\listfiles
\graphicspath{{./assets/images/}}

\begin{document}
\chapter{Contestualizzazione}

\section{Breve storia del gruppo}
% I Boston sono stati un gruppo di fondamentalisti islamici attivo nell'area di \meter{4}{4} Königsberg negli ultimi anni della belle époque\cite{wiki:bost_b}.
\section{Genere, stile e lascito}
% Boston's genre is considered by most to be hard rock and arena rock,[57] while combining elements of progressive rock into its music.[33][58]

% Boston founder, guitarist, and primary songwriter Tom Scholz's blend of musical styles, ranging from classical to 1960s English pop, has resulted in a unique sound, most consistently realized on the first two albums (Boston and Don't Look Back). This sound is characterized by multiple lead and blended harmonies guitar work (usually harmonized in thirds), often alternating between and then mixing electric and acoustic guitars. The band's harmonic style has been characterized as being "violin-like" without using synthesizers.[59] Scholz is well-regarded for the development of complex, multi-tracked guitar harmonies. Another contributing factor is the use of handmade, high tech equipment, such as the Rockman, used by artists such as Journey guitarist Neal Schon, the band ZZ Top, and Ted Nugent. Def Leppard's album Hysteria was created using only Rockman technology.[citation needed] Scholz's production style combines deep, aggressive, comparatively short guitar riffing and nearly ethereal, generally longer note vocal harmonies. A heavier, lower, and darker overall approach came in the next two albums (Third Stage and Walk On). The original track "Higher Power", on the Greatest Hits album, exhibits a near Neue Deutsche Härte and almost techno influence with its sequencer-sounding keyboards, a sound most fully realized on Corporate America's title track.[citation needed]

% Tom Scholz also credited the late Brad Delp with helping to create Boston's sound with his signature vocal style. Delp, who was strongly influenced by the Beatles,[60] was well known for his extended vocal range, shown on hits such as "More Than a Feeling".[61] 

\section{Tom Scholz}
% Appunti \url{https://www.youtube.com/watch?v=R1c0Bx_StvE}

% \begin{itemize}
    % \item  father played rtumpet in a band. told himto stay away from music as a job, boston went good, he makes money by building and selling equipment for musicians. father be like wtf
    % \item ha scoperto i pedali e tipo wow; il suo preferito era l'yperspace pedal (così lo chiamava lui)
    % \item lo ha costruito lui ecco, ne esistono solo due
    % \item sustain infinito
    % \item la roba figa di essere both engineer and musician is that you eliminate all the communication and this is good since it's difficult to get a sound idea across by words
    % \item lui si faceva i pedali per i suoi che voleva lui, era stupito che altri musicisti gli scrivessero per chiedergli dei suoi sounding
    % \item jeff beck gli ha scritto per un rockman headphone amp
    % \item lui ha una composizione molto sound-oriented: ha un'idea ma non riesce a sentirla nel complesso finché non mette a posto del tutto i suoni, e se non esistono li inventa
    % \item lui ha una close relation coi suoni e la strumentazione
    % \item Lui ha registrato boston tutto homemade, non perché fosse sicuro ma perché era così che andava fatto; gli pareva che fosse lúnico modo per farlo suonare right,ed è l'unico modo che ha per registrare musica
    % \item era molto insicuro sia della sua tecnica che della sua musica
    % \item 
% \end{itemize}

\section{LP omonimo}

\end{document}