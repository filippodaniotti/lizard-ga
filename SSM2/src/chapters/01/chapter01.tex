\documentclass[class=book, crop=false, oneside, 12pt]{standalone}
\usepackage{standalone}
\usepackage{../../style}
\listfiles
\graphicspath{{./assets/images/}}

\begin{document}
\chapter{Contestualizzazione}
I Boston sono un gruppo rock americano fondato nell'omonima città del Massachussets nel 1975. A fronte di numerosi cambi di formazione, la costante presenza del leader polistrumentista Tom Scholz e del cantante Brad Delp (fino alla morte nel 2007) hanno permesso alla formazione di mantenere identità e stile caratteristico per oltre quattro decadi di attività piuttosto discontinua. 

Il complesso ha all'attivito sei album in studio e oltre 75 milioni di dischi venduti, 25 dei quali relativi al solo primo disco omonimo, che risulta pertanto uno dei dischi di debutto più venduti della storia della musica hard rock. 

\section{Tom Scholz e i Boston}
La storia e la direzione musicale dei Boston sono intimamente legati alla figura di Donald Thomas Scholz, fondatore nonché principale compositore del gruppo.

Scholz colloca l'inizio del percorso musicale nel 1969 quando, mentre frequentava il MIT, ha scritto una prima versione di un brano musicale che sarebbe stato successivamente noto come Foreplay\cite{wiki:bost_b}. Sempre all'interno dell'ambiente del MIT Scholz entrò in un gruppo di nome Freehold, all'interno del quale conobbe il chitarrista Barry Godreau e il batterista Jim Masdea. È con quest'ultimo che Scholz collaborò più intensamente negli anni a venire.

Dopo aver conseguito il Master's Degree in ingegneria meccanica, Scholz trovò lavoro presso il dipartimento di product design presso Polaroid; contemporaneamente, continuava a suonare come tastierista in numerose formazioni dell'area di Boston, continuando a collaborare con Masdea e Godreau e arrivando a fare la conoscenza del cantante Brad Delp. 

Parallelamente a questo, sviluppò assieme a Masdea un acceso interesse per la registrazione e la produzione audio; l'interessere e la curiosità dei due musicisti era tale che allestirono una sorta di studio di registrazione amatoriale nel seminterrato di Masdea, all'interno del quale iniziarono a registrare delle demo. Masdea si occupava delle tracce di batteria, mentre Scholz di tutto il resto. Tutto l'equipaggiamento che il duo utilizzava per registrare era autocostruito da Scholz, che mise a frutto gli anni di studio al MIT e l'esperienza nel dipartimento di Product Design a Polaroid per costruire un rudimentale registratore a nastri. 

Il perfezionismo e la morbosa ricerca dei suoni particolari che aveva in mente portarono Scholz a non limitarsi alla costruzione della strumentazione tecnica necessaria a registrare, ma progettò anche gran parte dell'equipaggiamento per la chiatarra, dagli amplificatori ai pedali effetto. In un'intervista ha dichiarato che per lui la ricerca dei suoni e la composizione dei brani sono due attività che procedono di pari passo, e proprio grazie al suo essere sia un ingegnere sia un musicista può eliminare le difficoltà di comunicazione che solitamente sussistono tra queste due figure e creare esattamente i suoni che gli servono per i brani che ha in mente\cite{yt:scholz_sound}.

Nel biennio 1973-1974 Scholz ingaggiò Brad Delp per registrare la voce dei brani fino ad allora registrati e ultimò un disco demo contente More Than a Feeling, Peace of Mind, Rock \& Roll Band, Hitch a Ride e Don't Look Back, che spedì a tutte le maggiori etichette discografiche del periodo. Tutte quante rifiutarono la demo e Scholz, che ormai aveva quasi 28, considerava fondamentamentalmente conclusa la sua esperienza musicale.

Per una serie di fortunate coincidenze\cite{wiki:bost_a}, la demo di Scholz catturò l'attenzione di Paul Ahern e Charles McKenzie\cite{wiki:bost_b}, che proposero a Scholz un contratto con Epic Records. In quel periodo Masdea perse interesse nel progetto e lasciò la formazione poco prima che Scholz e Delp firmassero il contratto. La formazione venne dunque completata con Braay Godreau alla chitarra solista, Fran Sheehan al basso e Sib Hashian alla batteria.

Ci fu un breve screzio tra il produttore Scholz e il produttore John Boylan: questi avrebbe preferito infatti che tutte le tracce del disco fossero ri-registrate in uno studio professionale di Epic a Los Angeles, ma Scholz era assolutamente contrario all'idea di abbandonare la sua strumentazione e l'ambiente del seminterrato in cui aveva registrato le demo fino a quel momento. Raggiunsero quindi un compromesso: Scholz avrebbe ri-registrato le tracce di sua competenza (ovvero tuttei gli organi, i clavinet e la maggior parte delle chitarre e bassi) nel suo seminterrato, mentre il resto del gruppo avrebbe registrato nello studio Epic. A fine registrazione, le tracce sarebbero state quindi mandate a Boylan per procedere al mix e al mastering.

L'album venne rilasciato infine il 25 agosto 1976 e fu subito un incredibile successo, arrivando a vendere 17 milioni di copie nei soli Stati Uniti e toccando la posizione numero 3 nella Billboard 200 e restando in classifica per 132 settimane\cite{wiki:bost_b}. Vennero rilasciati tre singoli: More Than a Feeling, Long Time e Peace of Mind.

Il Successivo disco \emph{Don't Look Back} uscì nel 1978. Nonostante un intervello di due anni tra LP fosse considerato piuttosto ampio a quel tempo, Scholz accusò le pressioni della Epic e si dichiarò insoddisfatto del risultato. Il disco ebbe comunque un discreto successo, riuscindo a vendere oltre 7 milioni di copie.

A seguito di questo l'attività del gruppo divenne molto più sparsa e sporadica. Dal 1979 fino a oltre la metà degli anni '80 i membri del gruppo si dedicarono a progetti indivuali. In particolare, Scholz si dedicò alla progettazione e vendita di strumentazione per chitarra attraverso la sua compagnia \emph{Scholz Research \& Development, Inc.}, fondata nel 1980\cite{wiki:scholz_inc}; di grande successo fu in particolare la linea di amplificatori e processori di segnale Rockman, che registro buoni dati di vendita per tutti gli anni '80. La compagnia sopravvisse fino al 1995, quando Scholz stesso la chiuse e vendette la linea Rockman alla Dunlop Manufacturing, Inc\cite{wiki:scholz_inc}. 

Nonostante gli album rilasciati successivamente a questo periodo non riuscirono mai a replicare il successo prorompente ottenuto nei primi anni dalla band, Scholz mantenne piuttosto costantemente l'attività musicale. In questo modo abbiamo il rilascio di \emph{Third Stage} nel 1986, \emph{Walk On} nel 1994, un \emph{Greatest Hits} nel 1997 e \emph{Corporate America} nel 2002.

Il gruppo ricevette una profonda scossa nel 2007, quando il cantante Brad Delp si suicidò nella propria abitazione tramite l'inalazione di monossido di carbonio. Il 17 agosto dello stesso anno venne organizzato un concerto tributo a lui dedicato alla Bank of America Pavilon a Boston.

A seguito di questo, il gruppo proseguì una intermittente attività di touring. Nel 2013 venne rilasciato il piu recente disco a nome Boston, \emph{Live, Love \& Hope}, che ancora conteneva delle tracce vocali registrate da Delp. Nel 2017 Scholz ha annunciato un nuovo album in lavorazione, senza azzardare possibili date di uscita.

\section{Genere, stile e lascito}
Lo stile dei Boston e il loro caratteristico suono trovano la loro più completa incarnazione nei primi due dischi della band, e in particolare nel primo; a maglie larghe, possiamo definire il loro genere come un hard rock misto ad arena rock\cite{book:rocknamerica}, con alcuni elementi tipici del progressive rock dell'epoca.

\subsection{Il Boston sound}
Il cosiddetto "Boston sound" è infatti una sintesi delle numerose influenze musicali di Tom Scholz, che vanno dal repertorio classico che il musicista ha frequentato da giovane durante gli studi di pianoforte classico, alla musica leggera degli anni '50, fino ad arrivare al rock americano e al pop britannico degli anni '60.

Tra gli elementi distintivi di questo suono troviamo una grande varietà timbrica di chitarre elettriche, che vengono impiegate sia in pulito sia in distorto, scolpite da pesante equalizzazione oppure contornate da peculiari modulazioni, rese possibili dalle abilità ingegneristiche di Scholz. Caratteristico è anche l'uso di chitarre elettriche in alternanza e, spesso, in parallo a quello delle chitarre acustiche (si senta ad esempio \emph{Peace of Mind} e \emph{Long Time}). Inoltre, la quasi totalità delle parti registrate nei dischi veniva registrata perlomeno due volte.

Dal mondo della musica classica Scholz ha invece ripreso l'uso delle armonizzazioni, soprattutto per terze (si senta l'assolo di \emph{More Than a Feeling} o \emph{Peace of Mind}), non molto comuni fino a quel momento. Sempre in \emph{More Than a Feeling}  abbiamo anche l'uso dei legati per rendere i mordenti. 

L'uso dell'organo Hammond B2 appartiene a entrambi i mondi: Scholz riesce infatti a utilizzarlo sia seguendo gli stilemi tipici del rock/blues, quindi con configurazione \(8880000\), leslie e ghost notes (\emph{Hitch a Ride}, \emph{Smokin'}), sia con un timbro più classico (sezione centrale di \emph{Smokin'}, \emph{Foreplay}). 

Tom Scholz era inoltre convinto che gran parte della riuscita del Boston sound fosse da attribuire alla voce di Brad Delp\cite{wiki:bost_b}: il suo stile di ispirazione Beatles, il suo imponente intervallo vocale e il timbro tanto potente quanto limpido e cristallino, a tratti angelico, creavano un perfetto contrasto il fraseggio deciso e aggressivo di Scholz.

\subsection{Lascito}
Ci sono numerosi elementi del Boston sound che sono stati ripresi negli anni, più o meno estensivamente. L'uso del multi-trackingè diventato uno standard dalle produzioni ad ampio budget fino a quelle più piccole e, con poche eccezioni motivate da ragioni stilistiche (ad esempio il movimento punk di fine anni '70), è tutt'oggi parte integrante del processo.

L'uso di armonizzazioni per terze maggiori verrà riutilizzato a più riprese, soprattutto in certi sottogeneri del metal di ispirazione classica (si pensi allo stile di Yngwie Malmsteen o al tipico suono iperprodotto del power metal). Anche l'uso del pick scratching diventerà un tratto tipico del suono chitarristico dell'hard rock degli anni '80 e fino ai giorni nostri (si pensi alla celeberrima introduzione di \emph{Hysteria} dei Muse).

\subsection{Il fattore home studio}
Tuttavia io sono convinto che l'elemento più notevole del Boston sound risieda nel fatto che si tratta quasi interamente di una produzione amatoriale. Si tratta di una situazione senza alcun vero precedente: non era mai successo che un singolo musicista riuscisse, peraltro a tempo perso nel proprio seminterrato, a ingegnerizzare la composizione e la registrazione di un disco a un livello tale da essere accettato e pubblicato come una qualsiasi altra produzione professionale. 

Si tratta di un risultato incredibile, nonostante possa apparire non così strano agli occhi di chi lo guarda distrattamente nel 2021 in retrospettiva. Questo perché noi ormai siamo abituati alla possibilità di registrare e produrre la nostra musica in casa. I computer ci forniscono un'immensa potenza di calcolo, con cui noi utilizziamo per far girare la nostra DAW preferita; questa, con la sua interfaccia intuitiva, ci fornisce un mixer e un sequencer dalle potenzialità sosstanzialmente infinite, e tutto quello che dobbiamo fare noi è caricare il nostro plugin VST preferito e iniziare a lavorare. E in tutto questo abbiamo sempre la consapevolezza che, assumendo di averne le conoscenze e competenze necessarie, saremo sempre in grado di ottenere un risultato meritevole di essere pubblicato. 

Al giorno d'oggi la scena della musica derivata dall'home studio rappresenta una porzione considerevole del totale delle produzioni. La maggior parte dei produttori della scena trap, anche in Italia (Sick Luke, Charlie Charles), sono nati in home studio. Se alcuni dovessero (forse a ragione) considerare la trap un genere senza grande valore musicale, si pensi allora a \emph{Somebody That I Used to Know}, famosissimo singolo del 2012 dell'artista belga Gotye e registrato in un fienile in una tenuta in Australia. Oppure ancora a qualsiasi produzione del chitarrista prog metal australiano Plini, che riporta la dicitura "composto, arrangiato, prodotto e mixato da Plini in una cameretta a Sydney"\cite{yt:plini}. Oppure all'attore più importante di tutta questa scena: Billie Eilish, che con un progetto musicale esclusivamente condotto da lei e dal fratello è riuscita a produrre musica di grando impatto e notevole valore musicale, vincere un Grammy Award e arrivare a vendere oltre 65 milioni di dischi prima ancora di aver compiuto vent'anni.

Tom Scholz è riuscito in questo prima che diventasse alla portata di tutti. Ha saputo capitalizzare sull'ottima istruzione ricevuta, sia accademica (MIT) sia musicale (studi classici), ed è riuscito a diventare un produttore in home studio ante litteram, forse unico precedente storico di quella che oggi è la normalità dell'industria musicale. 

Il fatto che la maggior parte della stessa strumentazione impiegata nella produzione fosse autocostruita, dal registratore all'effettistica per chitarra, non fa che aumentare la stima che, a fronte di tutto questo, io sento sia doveroso nutrire nei confronti di questo brillante musicista. 

\section{LP omonimo}

    % \item  father played rtumpet in a band. told himto stay away from music as a job, boston went good, he makes money by building and selling equipment for musicians. father be like wtf
    % \item ha scoperto i pedali e tipo wow; il suo preferito era l'yperspace pedal (così lo chiamava lui)
    % \item lo ha costruito lui ecco, ne esistono solo due
    % \item sustain infinito
    % \item la roba figa di essere both engineer and musician is that you eliminate all the communication and this is good since it's difficult to get a sound idea across by words
    % \item lui si faceva i pedali per i suoi che voleva lui, era stupito che altri musicisti gli scrivessero per chiedergli dei suoi sounding
    % \item jeff beck gli ha scritto per un rockman headphone amp
    % \item lui ha una composizione molto sound-oriented: ha un'idea ma non riesce a sentirla nel complesso finché non mette a posto del tutto i suoni, e se non esistono li inventa
    % \item lui ha una close relation coi suoni e la strumentazione
    % \item Lui ha registrato boston tutto homemade, non perché fosse sicuro ma perché era così che andava fatto; gli pareva che fosse lúnico modo per farlo suonare right,ed è l'unico modo che ha per registrare musica
    % \item era molto insicuro sia della sua tecnica che della sua musica
    % \item 


\end{document}