\documentclass[class=book, crop=false, oneside, 12pt]{standalone}
\usepackage{standalone}
\usepackage{../../style}
\graphicspath{{./assets/images/}}

\begin{document}
\chapter*{Considerazioni finali}
\addcontentsline{toc}{chapter}{Considerazioni finali}
Ho conosciuto i Boston per la prima volta quando ero ancora agli inizi del mio percorso musicale, quando ho sentito una cover di More Than a Feeling suonata in un saggio concerto da uno dei gruppi di musica d'insieme della scuola che frequentavo all'epoca. Come mio solito quando in quel periodo scoprivo un nuovo artista, sono andato a leggere informazioni su internet e ascoltare il disco di maggior successo del gruppo in questione, in questo caso ovviamente l'LP omonimo. 

All'epoca ero rimasto colpito in particolare dalle parti di organo, dal momento che le parti soliste erano numerose, tenevano l'organo bene in evidenza nel mix e suonavano "genuinamente rock",nonché molto appariscenti; era abbastanza per scaturire l'interesse di chiunque avesse il bias del tastierista. Riuscire a suonare Foreplay e l'assolo di Smokin' era diventato un obiettivo da perseguire.

Ma oltre a quello, già ero riuscito a intuire ed apprezzare l'enorme lavoro e cura dei dettagli che Tom Scholz aveva messo nel suo prodotto. Avevo letto infatti, seppur superficialmente, che gran parte del disco era stato  registrato in un seminterrato. All'epoca questo mi appariva come niente più che un particolare curioso e singolare, ma già tendevo ad accostare il suono delle chitarre del primo disco dei Boston molto più a quello dei gruppi hard rock/hair metal anni '80 che non a quello degli altri gruppi di metà anni '70 che conoscevo e ascoltavo, come ad esempio i Queen, i Deep Purple o il movimento progressive rock inglese e italiano.

\paragraph{}
Dopo aver approfondito la storia di Tom Scholz e aver dato uno sguardo in profondità a due tra le più rappresentative tracce del primo disco ho trovato conferma di tutte queste vecchie impressioni. 

Il suono dei Boston è decisamente diverso da qualsiasi altra produzione degli anni '70 e da solo potrebbe costituire l'identità della  band. Non è così, naturalmente: Scholz aggiunge alla sua offerta dei brani arrangiati nei minimi dettagli, forte di una tendenza estrema al perfezionismo che gli avrebbe causato, in futuro, dei problemi con le varie etichette discografiche.

Anche la composizione dei brani è estremamente ispirata e rappresenta un'eccellente sintesi dello stile del classic rock, con tutti i suoi canoni (scelta di strumenti, fraseggio blues, utilizzo del modo misolidio), con la musica classica che ha fatto parte dell'infanzia e dell'educazione musicale di Scholz; conferire a ogni singolo brano una propria identità non è certo un'impresa da tutti.

\paragraph{}
Credo sia proprio questo ciò che farò mio dell'esperienza di Tom Scholz e dei Boston. In un certo senso, Scholz ha saputo mettere tutto sé stesso nella sua musica: ha messo tutti i suoi gusti e le sue influenze musicali, come chiunque, ma ha saputo anche applicare la sua indole perfezionista che non ha mai barattato, nemmeno di fronte alla pressione delle case discografiche; senza di questa, la musica dei Boston sarebbe risultata considerevolmente diversa.

E infine, fattore più importante. Scholz ha saputo capitalizzare sulle conoscenze derivate dalla sua istruzione e sulle competenze acquisite nella carriera lavorativa, incanalando tutto questo a una grande perseveranza e determinazione con il fine di dare vita a un progetto che aveva in mente. Non ha mai avuto come obiettivo primario il successo: aveva semplicemente un progetto musicale in mente, sapeva di avere abbastanza competenze per portarlo alla luce e ha agito di conseguenza. Il fatto che il progetto effettivamente riuscisse o meno aveva un'improtanza relativa, Scholz aveva già considerato l'ipotesi di abbandonare il mondo della musica e ha affermato di non aver mai ritenuto possibile che la sua musica potesse piacere a un tale numero di persone.

Il progetto, in un ultima istanza, è stato effettivamente un successo; ma questo successo non è mai stato ricercato, è semplicemente la naturale conseguenza dell'immenso e ragionato lavoro di Scholz, dalla sua perseveranza e dalla genuina bontà delle sue idee. La storia di Tom Scholz è quella di un individuo brillante e umile che non crede nelle scuse e che deve il suo successo alla fatica e all'impegno profusi prima di qualsiasi altra cosa, e per questo motivo io ritengo sia una figura verso cui chiunque, musicista e non, potrebbe nutrire una stima profonda e sincera. 

\end{document}